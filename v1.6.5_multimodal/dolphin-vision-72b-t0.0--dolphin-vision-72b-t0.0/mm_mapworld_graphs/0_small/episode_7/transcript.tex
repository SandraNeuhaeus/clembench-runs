
\documentclass{article}
\usepackage{colortbl}
\usepackage{makecell}
\usepackage{multirow}
\usepackage{supertabular}

\begin{document}

\newcounter{utterance}

\twocolumn

{ \footnotesize  \setcounter{utterance}{1}
\setlength{\tabcolsep}{0pt}
\begin{supertabular}{c@{$\;$}|p{.15\linewidth}@{}p{.15\linewidth}p{.15\linewidth}p{.15\linewidth}p{.15\linewidth}p{.15\linewidth}}

    \# & $\;$A & \multicolumn{4}{c}{Game Master} & $\;\:$B\\
    \hline 

    \theutterance \stepcounter{utterance}  

    & & & \multicolumn{2}{p{0.3\linewidth}}{\cellcolor[rgb]{0.95,0.95,0.95}{%
	\makecell[{{p{\linewidth}}}]{% 
	  \tt {\tiny [GM$|$GM]}  
	 {"image": ["games/mm_mapworld/resources/images/ADE_train_00010897.jpg"]} 
	  } 
	   } 
	   } 
	 & & \\ 
 

    \theutterance \stepcounter{utterance}  

    & & \multicolumn{4}{p{0.6\linewidth}}{\cellcolor[rgb]{0.9,0.9,0.9}{%
	\makecell[{{p{\linewidth}}}]{% 
	  \tt {\tiny [A$\langle$GM]}  
	 {"start": [1, 2], "size": 4, "game": "mm_mapworld_graphs"} 
	  } 
	   } 
	   } 
	 & \\ 
 

    \theutterance \stepcounter{utterance}  

    & \multicolumn{4}{p{0.6\linewidth}}{\cellcolor[rgb]{0.8,1,0.9}{%
	\makecell[{{p{\linewidth}}}]{% 
	  \tt {\tiny [A$\rangle$GM]}  
	 We are currently in this room. Please help me with the following task. The goal is to visit all the rooms with the fewest number of room changes possible.  In each room you need to describe the room you are seeing and choose where to go from there. Additionally, you need to provide a graph representing the map you have uncovered. Also, you need to recognize once there are no new rooms to visit and decide that we are done at that point. Please give your answer in the following format: \\ \tt '{"action":"<action>", "description": "<room description>", "graph": <graph>}'.\\ \tt <action> needs to be in the format "GO: <direction>" where <direction> is one of [north, east, south, west]. Alternatively, choose "DONE" as your action once you have explored the entire map.\\ \tt <room description> should be a single sentence describing the room shown to you.\\ \tt <graph> represents the map in this format: {"nodes":[], "edges":{"north": [], "south": [], "east": [], "west": []}}\\ \tt Omit any other text and answer only following the format, not adding anything except the dictionary!\\ \tt \\ \tt Here is an example: \\ \tt We are in this room. From here we can go: south, west. What is your next instruction?\\ \tt {"action":"GO: north", "description": "We are in a kitchen with a red fridge.", "graph":{"nodes":["Kitchen"], "edges":{"north": [], "south": [], "east": [], "west": []}}}\\ \tt We have made a step and are now in this room. From here we can go: east. What is your next instruction?\\ \tt {"action":"GO: east", "description": "We are in a living room with a couch and a tv.", "graph":{"nodes":["Kitchen", "Living Room"], "edges":{"north": [["Kitchen", "Living Room"]], "south": [], "east": [], "west": []}}}\\ \tt You have made a step and are now in this room. From here we can go: west, south. What is your next instruction?\\ \tt {"action":"GO: south", "description": "We are in a bedroom with two beds and a nightstand.",  "graph":{"nodes":["Kitchen", "Living Room", "Bedroom"], "edges":{"north": [["Kitchen", "Living Room"]], "south": [], "east": [["Living Room", "Bedroom"]], "west": []}}}\\ \tt ...\\ \tt You have made a step and are now in this room. From here we can go: north. What is your next instruction?\\ \tt Example answer:\\ \tt {"action":"DONE", "description": "We are in a stairwell, the stair is curved.", "graph":"{...}"}\\ \tt \\ \tt Let us start.\\ \tt Currently available directions: south. What is your next instruction?\\ \tt  
	  } 
	   } 
	   } 
	 & & \\ 
 

    \theutterance \stepcounter{utterance}  

    & & \multicolumn{4}{p{0.6\linewidth}}{\cellcolor[rgb]{0.9,0.9,0.9}{%
	\makecell[{{p{\linewidth}}}]{% 
	  \tt {\tiny [GM$\rangle$B]}  
	 We are currently in this room. Please help me with the following task. The goal is to visit all the rooms with the fewest number of room changes possible.  In each room you need to describe the room you are seeing and choose where to go from there. Additionally, you need to provide a graph representing the map you have uncovered. Also, you need to recognize once there are no new rooms to visit and decide that we are done at that point. Please give your answer in the following format: \\ \tt '{"action":"<action>", "description": "<room description>", "graph": <graph>}'.\\ \tt <action> needs to be in the format "GO: <direction>" where <direction> is one of [north, east, south, west]. Alternatively, choose "DONE" as your action once you have explored the entire map.\\ \tt <room description> should be a single sentence describing the room shown to you.\\ \tt <graph> represents the map in this format: {"nodes":[], "edges":{"north": [], "south": [], "east": [], "west": []}}\\ \tt Omit any other text and answer only following the format, not adding anything except the dictionary!\\ \tt \\ \tt Here is an example: \\ \tt We are in this room. From here we can go: south, west. What is your next instruction?\\ \tt {"action":"GO: north", "description": "We are in a kitchen with a red fridge.", "graph":{"nodes":["Kitchen"], "edges":{"north": [], "south": [], "east": [], "west": []}}}\\ \tt We have made a step and are now in this room. From here we can go: east. What is your next instruction?\\ \tt {"action":"GO: east", "description": "We are in a living room with a couch and a tv.", "graph":{"nodes":["Kitchen", "Living Room"], "edges":{"north": [["Kitchen", "Living Room"]], "south": [], "east": [], "west": []}}}\\ \tt You have made a step and are now in this room. From here we can go: west, south. What is your next instruction?\\ \tt {"action":"GO: south", "description": "We are in a bedroom with two beds and a nightstand.",  "graph":{"nodes":["Kitchen", "Living Room", "Bedroom"], "edges":{"north": [["Kitchen", "Living Room"]], "south": [], "east": [["Living Room", "Bedroom"]], "west": []}}}\\ \tt ...\\ \tt You have made a step and are now in this room. From here we can go: north. What is your next instruction?\\ \tt Example answer:\\ \tt {"action":"DONE", "description": "We are in a stairwell, the stair is curved.", "graph":"{...}"}\\ \tt \\ \tt Let us start.\\ \tt Currently available directions: south. What is your next instruction?\\ \tt  
	  } 
	   } 
	   } 
	 & \\ 
 

    \theutterance \stepcounter{utterance}  

    & & & \multicolumn{4}{p{0.6\linewidth}}{\cellcolor[rgb]{1,0.85,0.72}{%
	\makecell[{{p{\linewidth}}}]{% 
	  \tt {\tiny [GM$\langle$B]}  
	 Based on the provided image and the context of the question, there is no clear indication of available directions within the room, as there are no visible doors or paths indicating where one could go next. Therefore, it is not possible to give a direction in the format required (north, south, east, west). However, if we assume the perspective of the photograph is facing north and we are looking into the room from the entrance, the next likely step would be to turn and walk out the entrance to continue exploring. In this scenario, the action would be to go backward, which is not represented as an option in the provided instructions. Without additional context or a different perspective, it is not feasible to construct a graph or provide a descriptive action for the next step following the given format. If this were a game or virtual experience, additional information or visibility of different paths would be necessary to proceed. As such, the response would be:\\ \tt \\ \tt {"action":"NOT POSSIBLE", "description": "An error occurred, the available directions are not visible in the provided image.", "graph": {}} 
	  } 
	   } 
	   } 
	  \\ 
 

    \theutterance \stepcounter{utterance}  

    & & & \multicolumn{2}{p{0.3\linewidth}}{\cellcolor[rgb]{0.95,0.95,0.95}{%
	\makecell[{{p{\linewidth}}}]{% 
	  \tt {\tiny [GM$|$GM]}  
	 Game aborted. 
	  } 
	   } 
	   } 
	 & & \\ 
 

    \theutterance \stepcounter{utterance}  

    & & & \multicolumn{2}{p{0.3\linewidth}}{\cellcolor[rgb]{0.95,0.95,0.95}{%
	\makecell[{{p{\linewidth}}}]{% 
	  \tt {\tiny [GM$|$GM]}  
	 True 
	  } 
	   } 
	   } 
	 & & \\ 
 

\end{supertabular}
}

\end{document}
