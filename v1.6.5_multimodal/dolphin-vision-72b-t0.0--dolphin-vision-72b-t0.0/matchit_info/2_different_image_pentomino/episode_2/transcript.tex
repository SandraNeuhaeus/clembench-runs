
\documentclass{article}
\usepackage{colortbl}
\usepackage{makecell}
\usepackage{multirow}
\usepackage{supertabular}

\begin{document}

\newcounter{utterance}

\twocolumn

{ \footnotesize  \setcounter{utterance}{1}
\setlength{\tabcolsep}{0pt}
\begin{supertabular}{c@{$\;$}|p{.15\linewidth}@{}p{.15\linewidth}p{.15\linewidth}p{.15\linewidth}p{.15\linewidth}p{.15\linewidth}}

    \# & $\;$A & \multicolumn{4}{c}{Game Master} & $\;\:$B\\
    \hline 

    \theutterance \stepcounter{utterance}  

    & & \multicolumn{4}{p{0.6\linewidth}}{\cellcolor[rgb]{0.9,0.9,0.9}{%
	\makecell[{{p{\linewidth}}}]{% 
	  \tt {\tiny [A$\langle$GM]}  
	 You are participating in a collaborative guessing game.\\ \tt The goal is to find out whether this picture and another picture only I can see, are the same.\\ \tt Please describe your image first. Then, I will provide my description and we can ask each other questions about the images to figure out whether they are the same. Each of us can ask 3 questions. Now start your short image description with "DESCRIPTION:" followed by the description. Do not add anything else. 
	  } 
	   } 
	   } 
	 & \\ 
 

    \theutterance \stepcounter{utterance}  

    & \multicolumn{4}{p{0.6\linewidth}}{\cellcolor[rgb]{0.8,1,0.9}{%
	\makecell[{{p{\linewidth}}}]{% 
	  \tt {\tiny [A$\rangle$GM]}  
	 DESCRIPTION: The image is a pixel art-style graphic that includes the letters "P," "K," "E," "T," and an incomplete "Z," with a plus sign between "K" and "E." The letters are stylized with varied thicknesses and angles, and each letter and symbol has a distinct color: pink for "P," green for "K," yellow for "E," blue for "T," and orange for the bottom of "Z." The letters are not aligned and overlap in some places, creating an abstract, somewhat chaotic visual appearance. 
	  } 
	   } 
	   } 
	 & & \\ 
 

    \theutterance \stepcounter{utterance}  

    & & & \multicolumn{2}{p{0.3\linewidth}}{\cellcolor[rgb]{0.95,0.95,0.95}{%
	\makecell[{{p{\linewidth}}}]{% 
	  \tt {\tiny [GM$|$GM]}  
	 continue 
	  } 
	   } 
	   } 
	 & & \\ 
 

    \theutterance \stepcounter{utterance}  

    & & \multicolumn{4}{p{0.6\linewidth}}{\cellcolor[rgb]{0.9,0.9,0.9}{%
	\makecell[{{p{\linewidth}}}]{% 
	  \tt {\tiny [GM$\rangle$B]}  
	 You are participating in a collaborative guessing game.\\ \tt The goal is to find out whether this picture and another picture only I can see, are the same.\\ \tt Please describe your image first. Then, I will provide my description and we can ask each other questions about the images to figure out whether they are the same. Each of us can ask 3 questions. Now start your short image description with "DESCRIPTION:" followed by the description. Do not add anything else. 
	  } 
	   } 
	   } 
	 & \\ 
 

    \theutterance \stepcounter{utterance}  

    & & & \multicolumn{4}{p{0.6\linewidth}}{\cellcolor[rgb]{1,0.85,0.72}{%
	\makecell[{{p{\linewidth}}}]{% 
	  \tt {\tiny [GM$\langle$B]}  
	 DESCRIPTION: The image shows a collection of six distinct Tetris-like blocks, each of a different color and shape. The configuration seems random, and there is no apparent action or alignment among the blocks. Each block is isolated and not interacting with any other elements. The background is a solid white, with no patterns or shading, providing a stark contrast to the colorful blocks. 
	  } 
	   } 
	   } 
	  \\ 
 

    \theutterance \stepcounter{utterance}  

    & & & \multicolumn{2}{p{0.3\linewidth}}{\cellcolor[rgb]{0.95,0.95,0.95}{%
	\makecell[{{p{\linewidth}}}]{% 
	  \tt {\tiny [GM$|$GM]}  
	 continue 
	  } 
	   } 
	   } 
	 & & \\ 
 

    \theutterance \stepcounter{utterance}  

    & & \multicolumn{4}{p{0.6\linewidth}}{\cellcolor[rgb]{0.9,0.9,0.9}{%
	\makecell[{{p{\linewidth}}}]{% 
	  \tt {\tiny [GM$\rangle$B]}  
	 This is my DESCRIPTION: The image is a pixel art-style graphic that includes the letters "P," "K," "E," "T," and an incomplete "Z," with a plus sign between "K" and "E." The letters are stylized with varied thicknesses and angles, and each letter and symbol has a distinct color: pink for "P," green for "K," yellow for "E," blue for "T," and orange for the bottom of "Z." The letters are not aligned and overlap in some places, creating an abstract, somewhat chaotic visual appearance.\\ \tt Now ask a question in order to find out new aspects of my image that may be different to your image. Start with "QUESTION:" and do not add anything else. 
	  } 
	   } 
	   } 
	 & \\ 
 

    \theutterance \stepcounter{utterance}  

    & & & \multicolumn{4}{p{0.6\linewidth}}{\cellcolor[rgb]{1,0.85,0.72}{%
	\makecell[{{p{\linewidth}}}]{% 
	  \tt {\tiny [GM$\langle$B]}  
	 QUESTION: Can you describe the positioning of the plus sign in relation to the letters "P" and "Z"? 
	  } 
	   } 
	   } 
	  \\ 
 

    \theutterance \stepcounter{utterance}  

    & & & \multicolumn{2}{p{0.3\linewidth}}{\cellcolor[rgb]{0.95,0.95,0.95}{%
	\makecell[{{p{\linewidth}}}]{% 
	  \tt {\tiny [GM$|$GM]}  
	 continue 
	  } 
	   } 
	   } 
	 & & \\ 
 

    \theutterance \stepcounter{utterance}  

    & & \multicolumn{4}{p{0.6\linewidth}}{\cellcolor[rgb]{0.9,0.9,0.9}{%
	\makecell[{{p{\linewidth}}}]{% 
	  \tt {\tiny [A$\langle$GM]}  
	 This is my DESCRIPTION: The image shows a collection of six distinct Tetris-like blocks, each of a different color and shape. The configuration seems random, and there is no apparent action or alignment among the blocks. Each block is isolated and not interacting with any other elements. The background is a solid white, with no patterns or shading, providing a stark contrast to the colorful blocks.\\ \tt QUESTION: Can you describe the positioning of the plus sign in relation to the letters "P" and "Z"? Start your answer with "ANSWER:" and do not add anything else. 
	  } 
	   } 
	   } 
	 & \\ 
 

    \theutterance \stepcounter{utterance}  

    & \multicolumn{4}{p{0.6\linewidth}}{\cellcolor[rgb]{0.8,1,0.9}{%
	\makecell[{{p{\linewidth}}}]{% 
	  \tt {\tiny [A$\rangle$GM]}  
	 ANSWER: The plus sign is positioned between the letters "K" and the incomplete "Z," appearing slightly tilted to the left. It is not aligned with any baseline; rather, it seems to be at the midway point between the "E" and the "T" below it, creating a visual connection that ties the upper and lower halves of the composition together. 
	  } 
	   } 
	   } 
	 & & \\ 
 

    \theutterance \stepcounter{utterance}  

    & & & \multicolumn{2}{p{0.3\linewidth}}{\cellcolor[rgb]{0.95,0.95,0.95}{%
	\makecell[{{p{\linewidth}}}]{% 
	  \tt {\tiny [GM$|$GM]}  
	 continue 
	  } 
	   } 
	   } 
	 & & \\ 
 

    \theutterance \stepcounter{utterance}  

    & & \multicolumn{4}{p{0.6\linewidth}}{\cellcolor[rgb]{0.9,0.9,0.9}{%
	\makecell[{{p{\linewidth}}}]{% 
	  \tt {\tiny [A$\langle$GM]}  
	 Now ask a question in order to find out new aspects of my image that may be different to your image. Start with "QUESTION:" and do not add anything else. 
	  } 
	   } 
	   } 
	 & \\ 
 

    \theutterance \stepcounter{utterance}  

    & \multicolumn{4}{p{0.6\linewidth}}{\cellcolor[rgb]{0.8,1,0.9}{%
	\makecell[{{p{\linewidth}}}]{% 
	  \tt {\tiny [A$\rangle$GM]}  
	 Given your description of the Tetris-like blocks, they lack any form of connection or linking element such as a plus sign between them. Therefore, my question is:\\ \tt \\ \tt QUESTION: How are the Tetris-like blocks presented in terms of interaction? Are they in a grid-like structure that suggests gameplay, or are they floating or static in a non-playable arrangement? 
	  } 
	   } 
	   } 
	 & & \\ 
 

    \theutterance \stepcounter{utterance}  

    & & & \multicolumn{2}{p{0.3\linewidth}}{\cellcolor[rgb]{0.95,0.95,0.95}{%
	\makecell[{{p{\linewidth}}}]{% 
	  \tt {\tiny [GM$|$GM]}  
	 abort, first word: Given 
	  } 
	   } 
	   } 
	 & & \\ 
 

    \theutterance \stepcounter{utterance}  

    & & & \multicolumn{2}{p{0.3\linewidth}}{\cellcolor[rgb]{0.95,0.95,0.95}{%
	\makecell[{{p{\linewidth}}}]{% 
	  \tt {\tiny [GM$|$GM]}  
	 Aborted 
	  } 
	   } 
	   } 
	 & & \\ 
 

    \theutterance \stepcounter{utterance}  

    & & & \multicolumn{2}{p{0.3\linewidth}}{\cellcolor[rgb]{0.95,0.95,0.95}{%
	\makecell[{{p{\linewidth}}}]{% 
	  \tt {\tiny [GM$|$GM]}  
	 Aborted 
	  } 
	   } 
	   } 
	 & & \\ 
 

\end{supertabular}
}

\end{document}
