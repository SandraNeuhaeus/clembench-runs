
\documentclass{article}
\usepackage{colortbl}
\usepackage{makecell}
\usepackage{multirow}
\usepackage{supertabular}

\begin{document}

\newcounter{utterance}

\twocolumn

{ \footnotesize  \setcounter{utterance}{1}
\setlength{\tabcolsep}{0pt}
\begin{supertabular}{c@{$\;$}|p{.15\linewidth}@{}p{.15\linewidth}p{.15\linewidth}p{.15\linewidth}p{.15\linewidth}p{.15\linewidth}}

    \# & $\;$A & \multicolumn{4}{c}{Game Master} & $\;\:$B\\
    \hline 

    \theutterance \stepcounter{utterance}  

    & & \multicolumn{4}{p{0.6\linewidth}}{\cellcolor[rgb]{0.9,0.9,0.9}{%
	\makecell[{{p{\linewidth}}}]{% 
	  \tt {\tiny [A$\langle$GM]}  
	 {'role': 'user', 'content': 'You are given three images, one is called target and the other two are distractors.\nYour task is to generate a referring expression that best describes the target image while distinguishing it from the two other distractor images.\nThe first image is a distractor, the second image is a distractor, and the third image is the target. Instruction: Describe the target image. Generate the referring expression starting with the tag "Expression: " for the given target image. Omit any other text.', 'image': ['games/multimodal_referencegame/resources/clevr_images/27.jpg', 'games/multimodal_referencegame/resources/clevr_images/26.jpg', 'games/multimodal_referencegame/resources/clevr_images/25.jpg']} 
	  } 
	   } 
	   } 
	 & \\ 
 

    \theutterance \stepcounter{utterance}  

    & \multicolumn{4}{p{0.6\linewidth}}{\cellcolor[rgb]{0.8,1,0.9}{%
	\makecell[{{p{\linewidth}}}]{% 
	  \tt {\tiny [A$\rangle$GM]}  
	 To generate a referring expression that best describes the target image while distinguishing it from the two distractor images, we could use characteristics unique to the target. For example, if one of the distractors features a solid-colored background and the target does not, or if the shapes in the target image are differently colored from those in the distractors, these could be key distinguishing features. Unfortunately, without seeing the distractor images, I cannot create a precise referring expression that differentiates the target. If the target image is unique in its color scheme, shape arrangement, or background, any of these aspects could be used to distinguish it. For instance, a referring expression for a target image with a unique color scheme might start as "Expression: The image with the vibrant red and blue spheres against a stark white background." The exact referring expression should rely on the observable and distinct aspects of the target compared to the distractors. 
	  } 
	   } 
	   } 
	 & & \\ 
 

    \theutterance \stepcounter{utterance}  

    & & & \multicolumn{2}{p{0.3\linewidth}}{\cellcolor[rgb]{0.95,0.95,0.95}{%
	\makecell[{{p{\linewidth}}}]{% 
	  \tt {\tiny [GM$|$GM]}  
	 Invalid generated expression 
	  } 
	   } 
	   } 
	 & & \\ 
 

\end{supertabular}
}

\end{document}
