
\documentclass{article}
\usepackage{colortbl}
\usepackage{makecell}
\usepackage{multirow}
\usepackage{supertabular}

\begin{document}

\newcounter{utterance}

\twocolumn

{ \footnotesize  \setcounter{utterance}{1}
\setlength{\tabcolsep}{0pt}
\begin{supertabular}{c@{$\;$}|p{.15\linewidth}@{}p{.15\linewidth}p{.15\linewidth}p{.15\linewidth}p{.15\linewidth}p{.15\linewidth}}

    \# & $\;$A & \multicolumn{4}{c}{Game Master} & $\;\:$B\\
    \hline 

    \theutterance \stepcounter{utterance}  

    & & \multicolumn{4}{p{0.6\linewidth}}{\cellcolor[rgb]{0.9,0.9,0.9}{%
	\makecell[{{p{\linewidth}}}]{% 
	  \tt {\tiny [A$\langle$GM]}  
	 {'role': 'user', 'content': 'You are given three images, one is called target and the other two are distractors.\nYour task is to generate a referring expression that best describes the target image while distinguishing it from the two other distractor images.\nThe first image is a distractor, the second image is the target, and the third image is a distractor. Instruction: Describe the target image. Generate the referring expression starting with the tag "Expression: " for the given target image. Omit any other text.', 'image': ['games/multimodal_referencegame/resources/clevr_images/44.jpg', 'games/multimodal_referencegame/resources/clevr_images/32.jpg', 'games/multimodal_referencegame/resources/clevr_images/43.jpg']} 
	  } 
	   } 
	   } 
	 & \\ 
 

    \theutterance \stepcounter{utterance}  

    & \multicolumn{4}{p{0.6\linewidth}}{\cellcolor[rgb]{0.8,1,0.9}{%
	\makecell[{{p{\linewidth}}}]{% 
	  \tt {\tiny [A$\rangle$GM]}  
	 The task requires a referring expression that highlights the target image among the three provided. The referring expression should thus capture unique characteristics of the target image that are not shared with the distractors.\\ \tt \\ \tt To determine the unique features of the target image, we observe the three images and identify the elements that are distinctive to the target image. Once these distinguishing features are noted, they must be articulated into a coherent and specific referring expression.\\ \tt \\ \tt The target image, being distinct from the distractors, may exhibit unique colors, shapes, or thematic elements that are not present in the other two images. The referring expression should encapsulate these specific aspects, ensuring it clearly differentiates the target from the distractors.\\ \tt \\ \tt For example, if the target image has a unique color palette or features a particular subject not present in the distractors, this should be the focus of the referring expression. The expression should be clear and unambiguous, allowing the reader to easily identify the target image among the three options.\\ \tt \\ \tt In conclusion, the referring expression for the given target image should be crafted based on the unique elements that distinguish it from the distractor images, ensuring clarity and specificity in the identification. 
	  } 
	   } 
	   } 
	 & & \\ 
 

    \theutterance \stepcounter{utterance}  

    & & & \multicolumn{2}{p{0.3\linewidth}}{\cellcolor[rgb]{0.95,0.95,0.95}{%
	\makecell[{{p{\linewidth}}}]{% 
	  \tt {\tiny [GM$|$GM]}  
	 Invalid generated expression 
	  } 
	   } 
	   } 
	 & & \\ 
 

\end{supertabular}
}

\end{document}
