
\documentclass{article}
\usepackage{colortbl}
\usepackage{makecell}
\usepackage{multirow}
\usepackage{supertabular}

\begin{document}

\newcounter{utterance}

\twocolumn

{ \footnotesize  \setcounter{utterance}{1}
\setlength{\tabcolsep}{0pt}
\begin{supertabular}{c@{$\;$}|p{.15\linewidth}@{}p{.15\linewidth}p{.15\linewidth}p{.15\linewidth}p{.15\linewidth}p{.15\linewidth}}

    \# & $\;$A & \multicolumn{4}{c}{Game Master} & $\;\:$B\\
    \hline 

    \theutterance \stepcounter{utterance}  

    & & \multicolumn{4}{p{0.6\linewidth}}{\cellcolor[rgb]{0.9,0.9,0.9}{%
	\makecell[{{p{\linewidth}}}]{% 
	  \tt {\tiny [A$\langle$GM]}  
	 {'role': 'user', 'content': 'You are given three images, one is called target and the other two are distractors.\nYour task is to generate a referring expression that best describes the target image while distinguishing it from the two other distractor images.\nThe first image is a distractor, the second image is the target, and the third image is a distractor. Instruction: Describe the target image. Generate the referring expression starting with the tag "Expression: " for the given target image. Omit any other text.', 'image': ['games/multimodal_referencegame/resources/clevr_images/40.jpg', 'games/multimodal_referencegame/resources/clevr_images/32.jpg', 'games/multimodal_referencegame/resources/clevr_images/39.jpg']} 
	  } 
	   } 
	   } 
	 & \\ 
 

    \theutterance \stepcounter{utterance}  

    & \multicolumn{4}{p{0.6\linewidth}}{\cellcolor[rgb]{0.8,1,0.9}{%
	\makecell[{{p{\linewidth}}}]{% 
	  \tt {\tiny [A$\rangle$GM]}  
	 To address the task of generating a referring expression for the target image that distinguishes it from the two distractor images, one would focus on the unique features of each image. For instance, the target image may have distinct geometric shapes or colors that are not present in the distractors. If the target image features a specific scene or elements such as an outdoor landscape or an urban setting, the expression would highlight these key features. Additionally, if the target image has a unique color palette, texture, or lighting, these aspects should be included in the referring expression. The goal is to create a concise and specific description that sets the target image apart from the distractors, ensuring that it can be easily identified based on the expression provided. 
	  } 
	   } 
	   } 
	 & & \\ 
 

    \theutterance \stepcounter{utterance}  

    & & & \multicolumn{2}{p{0.3\linewidth}}{\cellcolor[rgb]{0.95,0.95,0.95}{%
	\makecell[{{p{\linewidth}}}]{% 
	  \tt {\tiny [GM$|$GM]}  
	 Invalid generated expression 
	  } 
	   } 
	   } 
	 & & \\ 
 

\end{supertabular}
}

\end{document}
