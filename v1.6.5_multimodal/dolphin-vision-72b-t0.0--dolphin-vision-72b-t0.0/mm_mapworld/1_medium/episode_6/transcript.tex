
\documentclass{article}
\usepackage{colortbl}
\usepackage{makecell}
\usepackage{multirow}
\usepackage{supertabular}

\begin{document}

\newcounter{utterance}

\twocolumn

{ \footnotesize  \setcounter{utterance}{1}
\setlength{\tabcolsep}{0pt}
\begin{supertabular}{c@{$\;$}|p{.15\linewidth}@{}p{.15\linewidth}p{.15\linewidth}p{.15\linewidth}p{.15\linewidth}p{.15\linewidth}}

    \# & $\;$A & \multicolumn{4}{c}{Game Master} & $\;\:$B\\
    \hline 

    \theutterance \stepcounter{utterance}  

    & & & \multicolumn{2}{p{0.3\linewidth}}{\cellcolor[rgb]{0.95,0.95,0.95}{%
	\makecell[{{p{\linewidth}}}]{% 
	  \tt {\tiny [GM$|$GM]}  
	 {"image": ["games/mm_mapworld/resources/images/ADE_train_00009525.jpg"]} 
	  } 
	   } 
	   } 
	 & & \\ 
 

    \theutterance \stepcounter{utterance}  

    & & \multicolumn{4}{p{0.6\linewidth}}{\cellcolor[rgb]{0.9,0.9,0.9}{%
	\makecell[{{p{\linewidth}}}]{% 
	  \tt {\tiny [A$\langle$GM]}  
	 {"start": [2, 0], "size": 6, "game": "mm_mapworld"} 
	  } 
	   } 
	   } 
	 & \\ 
 

    \theutterance \stepcounter{utterance}  

    & \multicolumn{4}{p{0.6\linewidth}}{\cellcolor[rgb]{0.8,1,0.9}{%
	\makecell[{{p{\linewidth}}}]{% 
	  \tt {\tiny [A$\rangle$GM]}  
	 We are currently in this room. Please help me with the following task. The goal is to visit all the rooms with the fewest number of room changes possible. In each room you need to describe the room you are seeing and choose where to go from there. Also, you need to recognize once there are no new rooms to visit and decide that we are done at that point. Please give your answer in the following format: "{"description": "<room description>", "action": "<action>"}". Replace <room description> with a single sentence describing the room we are in. To move to a neighboring room, replace <action> with "GO: DIRECTION" where DIRECTION can be one of [north, south, east, west]. To stop the exploration, replace <action> with "DONE". Omit any other text.\\ \tt Here is an example:\\ \tt We are in this room. From here we can go: north, west. What is your next instruction?\\ \tt {"description": "We are in a kitchen with a red fridge.", "action": "GO: north"}\\ \tt We have made a step and are now in this room. From here we can go: south, east. What is your next instruction?\\ \tt {"description": "We are in a living room with a couch and a tv.", "action": "GO: east"}\\ \tt ...\\ \tt We have made a step and are now in this room. From here we can go: south, east. What is your next instruction?\\ \tt {"description": "We are in a bathroom", "action": "DONE"}\\ \tt Let us start. \\ \tt We have made a step and are now in this room. From here we can go: east, north, west. What is your next instruction? 
	  } 
	   } 
	   } 
	 & & \\ 
 

    \theutterance \stepcounter{utterance}  

    & & \multicolumn{4}{p{0.6\linewidth}}{\cellcolor[rgb]{0.9,0.9,0.9}{%
	\makecell[{{p{\linewidth}}}]{% 
	  \tt {\tiny [GM$\rangle$B]}  
	 We are currently in this room. Please help me with the following task. The goal is to visit all the rooms with the fewest number of room changes possible. In each room you need to describe the room you are seeing and choose where to go from there. Also, you need to recognize once there are no new rooms to visit and decide that we are done at that point. Please give your answer in the following format: "{"description": "<room description>", "action": "<action>"}". Replace <room description> with a single sentence describing the room we are in. To move to a neighboring room, replace <action> with "GO: DIRECTION" where DIRECTION can be one of [north, south, east, west]. To stop the exploration, replace <action> with "DONE". Omit any other text.\\ \tt Here is an example:\\ \tt We are in this room. From here we can go: north, west. What is your next instruction?\\ \tt {"description": "We are in a kitchen with a red fridge.", "action": "GO: north"}\\ \tt We have made a step and are now in this room. From here we can go: south, east. What is your next instruction?\\ \tt {"description": "We are in a living room with a couch and a tv.", "action": "GO: east"}\\ \tt ...\\ \tt We have made a step and are now in this room. From here we can go: south, east. What is your next instruction?\\ \tt {"description": "We are in a bathroom", "action": "DONE"}\\ \tt Let us start. \\ \tt We have made a step and are now in this room. From here we can go: east, north, west. What is your next instruction? 
	  } 
	   } 
	   } 
	 & \\ 
 

    \theutterance \stepcounter{utterance}  

    & & & \multicolumn{4}{p{0.6\linewidth}}{\cellcolor[rgb]{1,0.85,0.72}{%
	\makecell[{{p{\linewidth}}}]{% 
	  \tt {\tiny [GM$\langle$B]}  
	 The room is a hotel bathroom, sparsely decorated with a vintage-style CRT television and a mirror. Given the available directions, the action to take next should be based on the logical flow of moving through rooms in a space like this. Since bathrooms are typically not central rooms but situated off to the side or between main rooms, the most common direction to go after visiting a bathroom might be east or west, depending on the typical layout of the hotel.\\ \tt \\ \tt Considering the standard floor plan of a hotel, bathrooms are often located next to bedrooms or sleeping areas. Therefore, the next reasonable move would be to head west, if it leads to a bedroom or a private space, which is a logical sequence in hotel room designs. East could lead to a hallway or another communal space, which might be necessary to explore as well, but without additional context, west seems the more likely choice.\\ \tt \\ \tt Hence, the next instruction should be:\\ \tt {"description": "We are in a hotel bathroom with a vintage television.", "action": "GO: west"} 
	  } 
	   } 
	   } 
	  \\ 
 

    \theutterance \stepcounter{utterance}  

    & & & \multicolumn{2}{p{0.3\linewidth}}{\cellcolor[rgb]{0.95,0.95,0.95}{%
	\makecell[{{p{\linewidth}}}]{% 
	  \tt {\tiny [GM$|$GM]}  
	 Game aborted. 
	  } 
	   } 
	   } 
	 & & \\ 
 

    \theutterance \stepcounter{utterance}  

    & & & \multicolumn{2}{p{0.3\linewidth}}{\cellcolor[rgb]{0.95,0.95,0.95}{%
	\makecell[{{p{\linewidth}}}]{% 
	  \tt {\tiny [GM$|$GM]}  
	 True 
	  } 
	   } 
	   } 
	 & & \\ 
 

\end{supertabular}
}

\end{document}
