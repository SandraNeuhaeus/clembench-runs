
\documentclass{article}
\usepackage{colortbl}
\usepackage{makecell}
\usepackage{multirow}
\usepackage{supertabular}

\begin{document}

\newcounter{utterance}

\twocolumn

{ \footnotesize  \setcounter{utterance}{1}
\setlength{\tabcolsep}{0pt}
\begin{supertabular}{c@{$\;$}|p{.15\linewidth}@{}p{.15\linewidth}p{.15\linewidth}p{.15\linewidth}p{.15\linewidth}p{.15\linewidth}}

    \# & $\;$A & \multicolumn{4}{c}{Game Master} & $\;\:$B\\
    \hline

    \theutterance \stepcounter{utterance}

    & & \multicolumn{4}{p{0.6\linewidth}}{\cellcolor[rgb]{0.9,0.9,0.9}{%
	\makecell[{{p{\linewidth}}}]{% 
	  \tt {\tiny [A$\langle$GM]}  
	 Давайте выполним следующую задачу. Цель состоит в том, чтобы заполнить пустую сетку, которая выглядит следующим образом:\\ \tt \\ \tt ▢ ▢ ▢ ▢ ▢\\ \tt ▢ ▢ ▢ ▢ ▢\\ \tt ▢ ▢ ▢ ▢ ▢\\ \tt ▢ ▢ ▢ ▢ ▢\\ \tt ▢ ▢ ▢ ▢ ▢\\ \tt \\ \tt Заполненная сетка ниже имеет размер 5 на 5 и может выглядеть следующим образом:\\ \tt \\ \tt ▢ ▢ ▢ ▢ ▢\\ \tt ▢ ▢ E ▢ ▢\\ \tt ▢ ▢ ▢ ▢ ▢\\ \tt ▢ ▢ ▢ ▢ ▢\\ \tt X X X X X\\ \tt \\ \tt Я хочу, чтобы вы описали мне эту сетку шаг за шагом. Вам не нужно описывать пустые квадраты, которые обозначены знаком «▢». Опишите только расположение букв в сетке. Затем вы ждете, пока я скажу: «Какая ваша следующая команда?», а затем переходите к следующему шагу. При подаче команд учитывайте размер сетки. Когда вы все описали, вы говорите «ГОТОВО».\\ \tt \\ \tt Для заполненной сетки выше приведены примеры шагов.\\ \tt \\ \tt Какая ваша следующая команда?\\ \tt Команда: Поставьте букву E во второй строке, в третьем столбце.\\ \tt \\ \tt Какая ваша следующая команда?\\ \tt Команда: заполнить последнюю строку знаком X\\ \tt \\ \tt Какая ваша следующая команда?\\ \tt Команда: ГОТОВО\\ \tt \\ \tt Хорошо. Сделайте это для следующего примера, который представляет собой сетку 5 на 5.\\ \tt \\ \tt N ▢ ▢ ▢ N\\ \tt N N ▢ N N\\ \tt N ▢ N ▢ N\\ \tt N N ▢ N N\\ \tt N ▢ ▢ ▢ N\\ \tt \\ \tt Какая ваша следующая команда?\\ \tt  
	  } 
	   } 
	   } 
	 & \\ 


    \theutterance \stepcounter{utterance}

    & \multicolumn{4}{p{0.6\linewidth}}{\cellcolor[rgb]{0.8,1,0.9}{%
	\makecell[{{p{\linewidth}}}]{% 
	  \tt {\tiny [A$\rangle$GM]}  
	 Let's start filling the grid.\\ \tt \\ \tt I'll describe the grid step by step. I'll only describe the non-empty cells.\\ \tt \\ \tt Initially, the grid has the following letters:\\ \tt \\ \tt N in the top-left corner\\ \tt N in the top-right corner\\ \tt N in the second row, first column\\ \tt N in the second row, second column\\ \tt N in the second row, fourth column\\ \tt N in the second row, fifth column\\ \tt N in the third row, first column\\ \tt N in the third row, third 
	  } 
	   } 
	   } 
	 & & \\ 


    \theutterance \stepcounter{utterance}

    & & & \multicolumn{2}{p{0.3\linewidth}}{\cellcolor[rgb]{0.95,0.95,0.95}{%
	\makecell[{{p{\linewidth}}}]{% 
	  \tt {\tiny [GM$|$GM]}  
	 Invalid instruction format 
	  } 
	   } 
	   } 
	 & & \\ 


\end{supertabular}
}

\end{document}
