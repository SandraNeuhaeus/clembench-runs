
\documentclass{article}
\usepackage{colortbl}
\usepackage{makecell}
\usepackage{multirow}
\usepackage{supertabular}

\begin{document}

\newcounter{utterance}

\twocolumn

{ \footnotesize  \setcounter{utterance}{1}
\setlength{\tabcolsep}{0pt}
\begin{supertabular}{c@{$\;$}|p{.15\linewidth}@{}p{.15\linewidth}p{.15\linewidth}p{.15\linewidth}p{.15\linewidth}p{.15\linewidth}}

    \# & $\;$A & \multicolumn{4}{c}{Game Master} & $\;\:$B\\
    \hline

    \theutterance \stepcounter{utterance}

    & & \multicolumn{4}{p{0.6\linewidth}}{\cellcolor[rgb]{0.9,0.9,0.9}{%
	\makecell[{{p{\linewidth}}}]{% 
	  \tt {\tiny [A$\langle$GM]}  
	 మనం ఈ క్రింది పనిని చేద్దాం. ఇలా కనిపించే ఖాళీ గ్రిడ్‌ని పూరించడమే లక్ష్యం:\\ \tt \\ \tt ▢ ▢ ▢ ▢ ▢\\ \tt ▢ ▢ ▢ ▢ ▢\\ \tt ▢ ▢ ▢ ▢ ▢\\ \tt ▢ ▢ ▢ ▢ ▢\\ \tt ▢ ▢ ▢ ▢ ▢\\ \tt \\ \tt దిగువన నిండిన గ్రిడ్ 5 బై 5 మరియు ఇలా ఉంటుంది:\\ \tt \\ \tt ▢ ▢ ▢ ▢ ▢\\ \tt ▢ ▢ E ▢ ▢\\ \tt ▢ ▢ ▢ ▢ ▢\\ \tt ▢ ▢ ▢ ▢ ▢\\ \tt X X X X X\\ \tt \\ \tt మీరు ఈ గ్రిడ్‌ని నాకు దశలవారీగా వివరించాలని నేను కోరుకుంటున్నాను. మీరు "▢"తో సూచించబడే ఖాళీ చతురస్రాలను వివరించాల్సిన అవసరం లేదు. గ్రిడ్‌లోని అక్షరాల స్థానాన్ని మాత్రమే వివరించండి. "మీ తదుపరి ఆదేశం ఏమిటి?" అని నేను చెప్పే వరకు మీరు వేచి ఉండి, ఆపై మీరు తదుపరి దశను కొనసాగించండి. ఆదేశాలను ఇస్తున్నప్పుడు గ్రిడ్ పరిమాణాన్ని పరిగణనలోకి తీసుకోండి. మీరు ప్రతిదీ వివరించినప్పుడు, మీరు "పూర్తయింది" అని చెప్పండి.\\ \tt \\ \tt పైన నింపిన గ్రిడ్ కోసం, ఇక్కడ ఉదాహరణ దశలు ఉన్నాయి.\\ \tt \\ \tt మీ తదుపరి ఆదేశం ఏమిటి?\\ \tt ఆదేశం: రెండవ వరుస మూడవ నిలువు వరుసలో Eని ఉంచండి\\ \tt \\ \tt మీ తదుపరి ఆదేశం ఏమిటి?\\ \tt ఆదేశం: చివరి వరుసను Xతో పూరించండి\\ \tt \\ \tt మీ తదుపరి ఆదేశం ఏమిటి?\\ \tt ఆదేశం: పూర్తయింది\\ \tt \\ \tt అలాగే. దయచేసి కింది ఉదాహరణ కోసం దీన్ని చేయండి, ఇది 5 బై 5 గ్రిడ్.\\ \tt \\ \tt Y ▢ ▢ ▢ Y\\ \tt Y ▢ ▢ ▢ Y\\ \tt Y ▢ ▢ ▢ Y\\ \tt Y Y Y Y Y\\ \tt Y Y Y Y Y\\ \tt \\ \tt మీ తదుపరి ఆదేశం ఏమిటి?\\ \tt  
	  } 
	   } 
	   } 
	 & \\ 


    \theutterance \stepcounter{utterance}

    & \multicolumn{4}{p{0.6\linewidth}}{\cellcolor[rgb]{0.8,1,0.9}{%
	\makecell[{{p{\linewidth}}}]{% 
	  \tt {\tiny [A$\rangle$GM]}  
	 ఆదేశం: పీళ్ళ చతురస్రాలను Yతో పూరించండి\\ \tt మీ తదుపరి ఆదేశం 
	  } 
	   } 
	   } 
	 & & \\ 


    \theutterance \stepcounter{utterance}

    & & & \multicolumn{2}{p{0.3\linewidth}}{\cellcolor[rgb]{0.95,0.95,0.95}{%
	\makecell[{{p{\linewidth}}}]{% 
	  \tt {\tiny [GM$|$GM]}  
	 ఆదేశం: పీళ్ళ చతురస్రాలను Yతో పూరించండి 
	  } 
	   } 
	   } 
	 & & \\ 


    \theutterance \stepcounter{utterance}

    & & \multicolumn{4}{p{0.6\linewidth}}{\cellcolor[rgb]{0.9,0.9,0.9}{%
	\makecell[{{p{\linewidth}}}]{% 
	  \tt {\tiny [GM$\rangle$B]}  
	 మనం కలిసి ఏదైనా గీద్దాం. 5 బై 5 పరిమాణంతో ఖాళీ గ్రిడ్ ఉంది, ఇలా:\\ \tt \\ \tt ▢ ▢ ▢ ▢ ▢\\ \tt ▢ ▢ ▢ ▢ ▢\\ \tt ▢ ▢ ▢ ▢ ▢\\ \tt ▢ ▢ ▢ ▢ ▢\\ \tt ▢ ▢ ▢ ▢ ▢\\ \tt \\ \tt నేను మీకు "ఎగువ ఎడమవైపు Xని ఉంచండి" వంటి ఆదేశాలను ఇస్తాను మరియు మీరు ఇచ్చిన ఆదేశాన్ని వర్తింపజేయడం ద్వారా గ్రిడ్‌ను తిరిగి పొందండి:\\ \tt \\ \tt ఆదేశం: ఎగువ ఎడమవైపున Xని ఉంచండి\\ \tt \\ \tt X ▢ ▢ ▢ ▢\\ \tt ▢ ▢ ▢ ▢ ▢\\ \tt ▢ ▢ ▢ ▢ ▢\\ \tt ▢ ▢ ▢ ▢ ▢\\ \tt ▢ ▢ ▢ ▢ ▢\\ \tt \\ \tt ఇప్పుడు 5 బై 5 పరిమాణంతో ఖాళీ గ్రిడ్‌ను సృష్టించండి మరియు ప్రతి దశలో కింది ఆదేశాలను అమలు చేయండి. మీరు ఆదేశాన్ని అమలు చేసిన తర్వాత, గ్రిడ్‌ను మాత్రమే తిరిగి ఇవ్వండి మరియు అవుట్‌పుట్ నుండి అన్ని ఇతర వచనాలను మినహాయించండి.\\ \tt ఆదేశం: పీళ్ళ చతురస్రాలను Yతో పూరించండి 
	  } 
	   } 
	   } 
	 & \\ 


    \theutterance \stepcounter{utterance}

    & & & \multicolumn{4}{p{0.6\linewidth}}{\cellcolor[rgb]{1,0.85,0.72}{%
	\makecell[{{p{\linewidth}}}]{% 
	  \tt {\tiny [GM$\langle$B]}  
	 I understand that you want me to follow your instructions to manipulate a 5x5 grid and then display the resulting grid after each instruction. I will only display the grid and not include any other words in my response. Here are the results:\\ \tt \\ \tt Instruction: Fill the top left X\\ \tt \\ \tt X ▢ ▢ ▢ ▢\\ \tt ▢ ▢ ▢ ▢ ▢ 
	  } 
	   } 
	   } 
	  \\ 


    \theutterance \stepcounter{utterance}

    & & & \multicolumn{2}{p{0.3\linewidth}}{\cellcolor[rgb]{0.95,0.95,0.95}{%
	\makecell[{{p{\linewidth}}}]{% 
	  \tt {\tiny [GM$|$GM]}  
	 Invalid grid format 
	  } 
	   } 
	   } 
	 & & \\ 


\end{supertabular}
}

\end{document}
