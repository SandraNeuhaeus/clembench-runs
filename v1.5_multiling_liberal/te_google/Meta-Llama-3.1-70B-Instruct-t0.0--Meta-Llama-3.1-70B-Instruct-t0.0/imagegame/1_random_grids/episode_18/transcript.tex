
\documentclass{article}
\usepackage{colortbl}
\usepackage{makecell}
\usepackage{multirow}
\usepackage{supertabular}

\begin{document}

\newcounter{utterance}

\twocolumn

{ \footnotesize  \setcounter{utterance}{1}
\setlength{\tabcolsep}{0pt}
\begin{supertabular}{c@{$\;$}|p{.15\linewidth}@{}p{.15\linewidth}p{.15\linewidth}p{.15\linewidth}p{.15\linewidth}p{.15\linewidth}}

    \# & $\;$A & \multicolumn{4}{c}{Game Master} & $\;\:$B\\
    \hline 

    \theutterance \stepcounter{utterance}  

    & & \multicolumn{4}{p{0.6\linewidth}}{\cellcolor[rgb]{0.9,0.9,0.9}{%
	\makecell[{{p{\linewidth}}}]{% 
	  \tt {\tiny [A$\langle$GM]}  
	 మనం ఈ క్రింది పనిని చేద్దాం. ఇలా కనిపించే ఖాళీ గ్రిడ్‌ని పూరించడమే లక్ష్యం:\\ \tt \\ \tt ▢ ▢ ▢ ▢ ▢\\ \tt ▢ ▢ ▢ ▢ ▢\\ \tt ▢ ▢ ▢ ▢ ▢\\ \tt ▢ ▢ ▢ ▢ ▢\\ \tt ▢ ▢ ▢ ▢ ▢\\ \tt \\ \tt దిగువన నిండిన గ్రిడ్ 5 బై 5 మరియు ఇలా ఉంటుంది:\\ \tt \\ \tt ▢ ▢ ▢ ▢ ▢\\ \tt ▢ ▢ E ▢ ▢\\ \tt ▢ ▢ ▢ ▢ ▢\\ \tt ▢ ▢ ▢ ▢ ▢\\ \tt X X X X X\\ \tt \\ \tt మీరు ఈ గ్రిడ్‌ని నాకు దశలవారీగా వివరించాలని నేను కోరుకుంటున్నాను. మీరు "▢"తో సూచించబడే ఖాళీ చతురస్రాలను వివరించాల్సిన అవసరం లేదు. గ్రిడ్‌లోని అక్షరాల స్థానాన్ని మాత్రమే వివరించండి. "మీ తదుపరి ఆదేశం ఏమిటి?" అని నేను చెప్పే వరకు మీరు వేచి ఉండి, ఆపై మీరు తదుపరి దశను కొనసాగించండి. ఆదేశాలను ఇస్తున్నప్పుడు గ్రిడ్ పరిమాణాన్ని పరిగణనలోకి తీసుకోండి. మీరు ప్రతిదీ వివరించినప్పుడు, మీరు "పూర్తయింది" అని చెప్పండి.\\ \tt \\ \tt పైన నింపిన గ్రిడ్ కోసం, ఇక్కడ ఉదాహరణ దశలు ఉన్నాయి.\\ \tt \\ \tt మీ తదుపరి ఆదేశం ఏమిటి?\\ \tt ఆదేశం: రెండవ వరుస మూడవ నిలువు వరుసలో Eని ఉంచండి\\ \tt \\ \tt మీ తదుపరి ఆదేశం ఏమిటి?\\ \tt ఆదేశం: చివరి వరుసను Xతో పూరించండి\\ \tt \\ \tt మీ తదుపరి ఆదేశం ఏమిటి?\\ \tt ఆదేశం: పూర్తయింది\\ \tt \\ \tt అలాగే. దయచేసి కింది ఉదాహరణ కోసం దీన్ని చేయండి, ఇది 5 బై 5 గ్రిడ్.\\ \tt ▢ ▢ U ▢ ▢\\ \tt ▢ ▢ ▢ ▢ ▢\\ \tt ▢ U ▢ ▢ ▢\\ \tt U U U ▢ ▢\\ \tt U ▢ ▢ ▢ ▢\\ \tt మీ తదుపరి ఆదేశం ఏమిటి?\\ \tt  
	  } 
	   } 
	   } 
	 & \\ 
 

    \theutterance \stepcounter{utterance}  

    & \multicolumn{4}{p{0.6\linewidth}}{\cellcolor[rgb]{0.8,1,0.9}{%
	\makecell[{{p{\linewidth}}}]{% 
	  \tt {\tiny [A$\rangle$GM]}  
	 మొదటి వరుస మూడవ నిలువు వరుసలో Uని ఉంచండి 
	  } 
	   } 
	   } 
	 & & \\ 
 

    \theutterance \stepcounter{utterance}  

    & & & \multicolumn{2}{p{0.3\linewidth}}{\cellcolor[rgb]{0.95,0.95,0.95}{%
	\makecell[{{p{\linewidth}}}]{% 
	  \tt {\tiny [GM$|$GM]}  
	 Invalid instruction format 
	  } 
	   } 
	   } 
	 & & \\ 
 

\end{supertabular}
}

\end{document}
