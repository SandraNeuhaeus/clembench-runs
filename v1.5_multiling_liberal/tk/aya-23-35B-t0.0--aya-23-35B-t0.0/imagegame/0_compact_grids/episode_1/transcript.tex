
\documentclass{article}
\usepackage{colortbl}
\usepackage{makecell}
\usepackage{multirow}
\usepackage{supertabular}

\begin{document}

\newcounter{utterance}

\twocolumn

{ \footnotesize  \setcounter{utterance}{1}
\setlength{\tabcolsep}{0pt}
\begin{supertabular}{c@{$\;$}|p{.15\linewidth}@{}p{.15\linewidth}p{.15\linewidth}p{.15\linewidth}p{.15\linewidth}p{.15\linewidth}}

    \# & $\;$A & \multicolumn{4}{c}{Game Master} & $\;\:$B\\
    \hline 

    \theutterance \stepcounter{utterance}  

    & & \multicolumn{4}{p{0.6\linewidth}}{\cellcolor[rgb]{0.9,0.9,0.9}{%
	\makecell[{{p{\linewidth}}}]{% 
	  \tt {\tiny [A$\langle$GM]}  
	 Aşakdaky meseläni ýerine ýetireliň. Maksat, şuňa meňzeş boş gözenegi doldurmak:\\ \tt \\ \tt ▢ ▢ ▢ ▢ ▢\\ \tt ▢ ▢ ▢ ▢ ▢\\ \tt ▢ ▢ ▢ ▢ ▢\\ \tt ▢ ▢ ▢ ▢ ▢\\ \tt ▢ ▢ ▢ ▢ ▢\\ \tt \\ \tt Aşakda doldurylan gözenek 5-den 5-e çenli we şeýle bolup biler:\\ \tt \\ \tt ▢ ▢ ▢ ▢ ▢\\ \tt ▢ ▢ E ▢ ▢\\ \tt ▢ ▢ ▢ ▢ ▢\\ \tt ▢ ▢ ▢ ▢ ▢\\ \tt X X X X X\\ \tt \\ \tt Bu seti maňa ädimme-ädim suratlandyrmagyňyzy isleýärin. "▢" bilen görkezilen boş öýjükleri suratlandyrmagyň zerurlygy ýok. Gözenekdäki diñe harplaryň öýjüklerini suratlandyryň. Soňra "Indiki buýrugyñyz näme?" diýmegime garaşýarsyňyz, soň bolsa indiki ädimiňizi dowam etdirersiňiz. Görkezmeler berlende gözenegiň ululygyny göz öňünde tutuň. Hemme zady suratlandyranyňyzda "GUTARDY" diýýärsiňiz.\\ \tt \\ \tt Aboveokardaky doldurylan gözenek üçin mysal ädimleri.\\ \tt \\ \tt Indiki buýrugyñyz näme?\\ \tt Buýruk: E-ni ikinji hatar üçünji sütüne goýuň\\ \tt \\ \tt Indiki buýrugyñyz näme?\\ \tt Buýruk: Iň soňky hatary X bilen dolduryň\\ \tt \\ \tt Indiki buýrugyñyz näme?\\ \tt Buýruk: GUTARDY\\ \tt \\ \tt Bolýar. Aşakdaky mysal üçin muny 5-den 5-e çenli gözenek bilen ýerine ýetirmegiňizi haýyş edýäris.\\ \tt \\ \tt ▢ ▢ ▢ ▢ S\\ \tt S S S S S\\ \tt ▢ ▢ ▢ ▢ S\\ \tt ▢ ▢ ▢ ▢ S\\ \tt S S S S S\\ \tt \\ \tt Indiki buýrugyñyz näme?\\ \tt  
	  } 
	   } 
	   } 
	 & \\ 
 

    \theutterance \stepcounter{utterance}  

    & \multicolumn{4}{p{0.6\linewidth}}{\cellcolor[rgb]{0.8,1,0.9}{%
	\makecell[{{p{\linewidth}}}]{% 
	  \tt {\tiny [A$\rangle$GM]}  
	 Buýruk: S-ni ikinji hatar üçünji sütüne goýuň 
	  } 
	   } 
	   } 
	 & & \\ 
 

    \theutterance \stepcounter{utterance}  

    & & & \multicolumn{2}{p{0.3\linewidth}}{\cellcolor[rgb]{0.95,0.95,0.95}{%
	\makecell[{{p{\linewidth}}}]{% 
	  \tt {\tiny [GM$|$GM]}  
	 Buýruk: S-ni ikinji hatar üçünji sütüne goýuň 
	  } 
	   } 
	   } 
	 & & \\ 
 

    \theutterance \stepcounter{utterance}  

    & & \multicolumn{4}{p{0.6\linewidth}}{\cellcolor[rgb]{0.9,0.9,0.9}{%
	\makecell[{{p{\linewidth}}}]{% 
	  \tt {\tiny [GM$\rangle$B]}  
	 Geliň, bilelikde bir zat çekeliň. 5-den 5-e çenli ululykdaky boş gözenek bar:\\ \tt \\ \tt ▢ ▢ ▢ ▢ ▢\\ \tt ▢ ▢ ▢ ▢ ▢\\ \tt ▢ ▢ ▢ ▢ ▢\\ \tt ▢ ▢ ▢ ▢ ▢\\ \tt ▢ ▢ ▢ ▢ ▢\\ \tt \\ \tt Size "ýokarky çep tarapda X goýuň" ýaly buýruklar bererin, berlen buýrugy ulanyp, seti yzyna berersiňiz:\\ \tt \\ \tt Buýruk: ýokarky çep tarapda X goýuň\\ \tt \\ \tt X ▢ ▢ ▢ ▢\\ \tt ▢ ▢ ▢ ▢ ▢\\ \tt ▢ ▢ ▢ ▢ ▢\\ \tt ▢ ▢ ▢ ▢ ▢\\ \tt ▢ ▢ ▢ ▢ ▢\\ \tt \\ \tt Indi 5-den 5 ululykdaky boş gözenek dörediň we her ädimde aşakdaky buýruklary ýerine ýetiriň. Buýrugy ýerine ýetireniňizden soň, diňe gözenegi yzyna gaýtaryň we beýleki tekstleri çykyşdan çykaryň.\\ \tt Buýruk: S-ni ikinji hatar üçünji sütüne goýuň 
	  } 
	   } 
	   } 
	 & \\ 
 

    \theutterance \stepcounter{utterance}  

    & & & \multicolumn{4}{p{0.6\linewidth}}{\cellcolor[rgb]{1,0.85,0.72}{%
	\makecell[{{p{\linewidth}}}]{% 
	  \tt {\tiny [GM$\langle$B]}  
	 X ▢ ▢ ▢\\ \tt ▢ ▢ ▢ ▢ X\\ \tt ▢ ▢ ▢ ▢ ▢\\ \tt \\ \tt Buýruk: S-ni ikinji hatar üçünji sütüne goýuň\\ \tt \\ \tt X ▢ ▢\\ \tt ▢ ▢ X\\ \tt ▢ ▢ ▢\\ \tt \\ \tt Buýruk: S-ni ikinji hatar üçünji sütüne goýuň\\ \tt \\ \tt X ▢ 
	  } 
	   } 
	   } 
	  \\ 
 

    \theutterance \stepcounter{utterance}  

    & & & \multicolumn{2}{p{0.3\linewidth}}{\cellcolor[rgb]{0.95,0.95,0.95}{%
	\makecell[{{p{\linewidth}}}]{% 
	  \tt {\tiny [GM$|$GM]}  
	 Invalid grid format 
	  } 
	   } 
	   } 
	 & & \\ 
 

\end{supertabular}
}

\end{document}
