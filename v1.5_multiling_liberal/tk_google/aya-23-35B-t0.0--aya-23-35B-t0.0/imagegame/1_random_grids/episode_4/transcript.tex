
\documentclass{article}
\usepackage{colortbl}
\usepackage{makecell}
\usepackage{multirow}
\usepackage{supertabular}

\begin{document}

\newcounter{utterance}

\twocolumn

{ \footnotesize  \setcounter{utterance}{1}
\setlength{\tabcolsep}{0pt}
\begin{supertabular}{c@{$\;$}|p{.15\linewidth}@{}p{.15\linewidth}p{.15\linewidth}p{.15\linewidth}p{.15\linewidth}p{.15\linewidth}}

    \# & $\;$A & \multicolumn{4}{c}{Game Master} & $\;\:$B\\
    \hline 

    \theutterance \stepcounter{utterance}  

    & & \multicolumn{4}{p{0.6\linewidth}}{\cellcolor[rgb]{0.9,0.9,0.9}{%
	\makecell[{{p{\linewidth}}}]{% 
	  \tt {\tiny [A$\langle$GM]}  
	 Aşakdaky meseläni ýerine ýetireliň. Maksat şuňa meňzeş boş seti doldurmak:\\ \tt \\ \tt ▢ ▢ ▢ ▢ ▢\\ \tt ▢ ▢ ▢ ▢ ▢\\ \tt ▢ ▢ ▢ ▢ ▢\\ \tt ▢ ▢ ▢ ▢ ▢\\ \tt ▢ ▢ ▢ ▢ ▢\\ \tt \\ \tt Aşakda doldurylan gözenek 5-den 5-e çenli bolup biler we şeýle bolup biler:\\ \tt \\ \tt ▢ ▢ ▢ ▢ ▢\\ \tt ▢ ▢ E ▢ ▢\\ \tt ▢ ▢ ▢ ▢ ▢\\ \tt ▢ ▢ ▢ ▢ ▢\\ \tt X X X X X\\ \tt \\ \tt Bu seti maňa ädimme-ädim suratlandyrmagyňyzy isleýärin. "▢" bilen görkezilen boş meýdançalary suratlandyrmagyň zerurlygy ýok. Diňe setdäki harplaryň ýerleşişini suratlandyryň. Soňra "Indiki buýrugyňyz näme?" Diýmegime garaşýarsyňyz, indiki ädimiňizi dowam etdirersiňiz. Buýruklar berlende gözenegiň ululygyny göz öňünde tutuň. Hemme zady suratlandyranyňyzda "BERILDI" diýýärsiňiz.\\ \tt \\ \tt Aboveokardaky doldurylan gözenek üçin mysal ädimleri.\\ \tt \\ \tt Indiki buýrugyňyz näme?\\ \tt Buýruk: E-ni ikinji hatara üçünji sütüne goýuň\\ \tt \\ \tt Indiki buýrugyňyz näme?\\ \tt Buýruk: Iň soňky hatary X bilen dolduryň\\ \tt \\ \tt Indiki buýrugyňyz näme?\\ \tt Buýruk: BERILDI\\ \tt \\ \tt Bolýar. Aşakdaky mysal üçin muny 5-den 5-e çenli gözenek bilen ýerine ýetirmegiňizi haýyş edýäris.\\ \tt ▢ ▢ ▢ ▢ ▢\\ \tt R ▢ R ▢ ▢\\ \tt ▢ ▢ ▢ ▢ R\\ \tt R ▢ R R ▢\\ \tt R ▢ R ▢ ▢\\ \tt Indiki buýrugyňyz näme?\\ \tt  
	  } 
	   } 
	   } 
	 & \\ 
 

    \theutterance \stepcounter{utterance}  

    & \multicolumn{4}{p{0.6\linewidth}}{\cellcolor[rgb]{0.8,1,0.9}{%
	\makecell[{{p{\linewidth}}}]{% 
	  \tt {\tiny [A$\rangle$GM]}  
	 Buýruk: Iň soňky hatary X bilen dolduryň\\ \tt \\ \tt Indiki buýrugyňyz: BERILDI 
	  } 
	   } 
	   } 
	 & & \\ 
 

    \theutterance \stepcounter{utterance}  

    & & & \multicolumn{2}{p{0.3\linewidth}}{\cellcolor[rgb]{0.95,0.95,0.95}{%
	\makecell[{{p{\linewidth}}}]{% 
	  \tt {\tiny [GM$|$GM]}  
	 Buýruk: Iň soňky hatary X bilen dolduryň 
	  } 
	   } 
	   } 
	 & & \\ 
 

    \theutterance \stepcounter{utterance}  

    & & \multicolumn{4}{p{0.6\linewidth}}{\cellcolor[rgb]{0.9,0.9,0.9}{%
	\makecell[{{p{\linewidth}}}]{% 
	  \tt {\tiny [GM$\rangle$B]}  
	 Geliň bilelikde bir zat çekeliň. 5-den 5-e çenli ululykdaky boş gözenek bar:\\ \tt \\ \tt ▢ ▢ ▢ ▢ ▢\\ \tt ▢ ▢ ▢ ▢ ▢\\ \tt ▢ ▢ ▢ ▢ ▢\\ \tt ▢ ▢ ▢ ▢ ▢\\ \tt ▢ ▢ ▢ ▢ ▢\\ \tt \\ \tt Size "ýokarky çep tarapda X goýuň" ýaly buýruklary bererin, berlen buýrugy ulanyp, seti yzyna berersiňiz:\\ \tt \\ \tt Buýruk: ýokarky çep tarapda X goýuň\\ \tt \\ \tt X ▢ ▢ ▢ ▢\\ \tt ▢ ▢ ▢ ▢ ▢\\ \tt ▢ ▢ ▢ ▢ ▢\\ \tt ▢ ▢ ▢ ▢ ▢\\ \tt ▢ ▢ ▢ ▢ ▢\\ \tt \\ \tt Indi 5-den 5 ululykdaky boş gözenek dörediň we her ädimde aşakdaky buýruklary ýerine ýetiriň. Buýrugy ýerine ýetireniňizden soň, diňe gözenegi yzyna gaýtaryň we beýleki tekstleri çykyşdan çykaryň.\\ \tt Buýruk: Iň soňky hatary X bilen dolduryň 
	  } 
	   } 
	   } 
	 & \\ 
 

    \theutterance \stepcounter{utterance}  

    & & & \multicolumn{4}{p{0.6\linewidth}}{\cellcolor[rgb]{1,0.85,0.72}{%
	\makecell[{{p{\linewidth}}}]{% 
	  \tt {\tiny [GM$\langle$B]}  
	 X ■ ■ ■ ■\\ \tt ■ ■ ■ ■ ■\\ \tt ■ ■ ■ ■ ■\\ \tt ■ ■ ■ ■ ■\\ \tt ■ ■ ■ ■ ■\\ \tt \\ \tt Buýruk: Iň soňky hatary X bilen dolduryň\\ \tt \\ \tt X X X X X X X X X X X X X X X X X X X X X X X X X X X X X X 
	  } 
	   } 
	   } 
	  \\ 
 

    \theutterance \stepcounter{utterance}  

    & & & \multicolumn{2}{p{0.3\linewidth}}{\cellcolor[rgb]{0.95,0.95,0.95}{%
	\makecell[{{p{\linewidth}}}]{% 
	  \tt {\tiny [GM$|$GM]}  
	 Invalid grid format 
	  } 
	   } 
	   } 
	 & & \\ 
 

\end{supertabular}
}

\end{document}
