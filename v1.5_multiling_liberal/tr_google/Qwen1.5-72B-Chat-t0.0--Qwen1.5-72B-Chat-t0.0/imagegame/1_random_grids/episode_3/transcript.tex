
\documentclass{article}
\usepackage{colortbl}
\usepackage{makecell}
\usepackage{multirow}
\usepackage{supertabular}

\begin{document}

\newcounter{utterance}

\twocolumn

{ \footnotesize  \setcounter{utterance}{1}
\setlength{\tabcolsep}{0pt}
\begin{supertabular}{c@{$\;$}|p{.15\linewidth}@{}p{.15\linewidth}p{.15\linewidth}p{.15\linewidth}p{.15\linewidth}p{.15\linewidth}}

    \# & $\;$A & \multicolumn{4}{c}{Game Master} & $\;\:$B\\
    \hline 

    \theutterance \stepcounter{utterance}  

    & & \multicolumn{4}{p{0.6\linewidth}}{\cellcolor[rgb]{0.9,0.9,0.9}{%
	\makecell[{{p{\linewidth}}}]{% 
	  \tt {\tiny [A$\langle$GM]}  
	 Aşağıdaki görevi yapalım. Amaç şuna benzeyen boş bir ızgarayı doldurmaktır:\\ \tt \\ \tt ▢ ▢ ▢ ▢ ▢\\ \tt ▢ ▢ ▢ ▢ ▢\\ \tt ▢ ▢ ▢ ▢ ▢\\ \tt ▢ ▢ ▢ ▢ ▢\\ \tt ▢ ▢ ▢ ▢ ▢\\ \tt \\ \tt Aşağıdaki dolu ızgara 5'e 5'tir ve şöyle görünebilir:\\ \tt \\ \tt ▢ ▢ ▢ ▢ ▢\\ \tt ▢ ▢ E ▢ ▢\\ \tt ▢ ▢ ▢ ▢ ▢\\ \tt ▢ ▢ ▢ ▢ ▢\\ \tt X X X X X\\ \tt \\ \tt Bana bu tabloyu adım adım anlatmanı istiyorum. "▢" ile gösterilen boş kareleri açıklamanıza gerek yok. Yalnızca ızgaradaki harflerin konumunu açıklayın. Daha sonra benim "Sıradaki komutunuz nedir?" dememi bekliyorsunuz ve bir sonraki adıma geçiyorsunuz. Komutları verirken ızgaranın boyutunu dikkate alın. Her şeyi anlattığınızda "BİTTİ" diyorsunuz.\\ \tt \\ \tt Yukarıdaki doldurulmuş ızgara için örnek adımları burada bulabilirsiniz.\\ \tt \\ \tt Bir sonraki komutunuz nedir?\\ \tt Komut: İkinci satırın üçüncü sütununa E koyun\\ \tt \\ \tt Bir sonraki komutunuz nedir?\\ \tt Komut: Son satırı X ile doldurun\\ \tt \\ \tt Bir sonraki komutunuz nedir?\\ \tt Komut: BİTTİ\\ \tt \\ \tt Tamam. Lütfen bunu 5'e 5'lik bir ızgara olan aşağıdaki örnek için yapın.\\ \tt ▢ ▢ ▢ Q ▢\\ \tt ▢ ▢ ▢ ▢ ▢\\ \tt ▢ ▢ ▢ ▢ ▢\\ \tt ▢ ▢ ▢ ▢ ▢\\ \tt ▢ ▢ Q Q Q\\ \tt Sıradaki komutunuz nedir?\\ \tt  
	  } 
	   } 
	   } 
	 & \\ 
 

    \theutterance \stepcounter{utterance}  

    & \multicolumn{4}{p{0.6\linewidth}}{\cellcolor[rgb]{0.8,1,0.9}{%
	\makecell[{{p{\linewidth}}}]{% 
	  \tt {\tiny [A$\rangle$GM]}  
	 Komut: İlk satırın ikinci sütununa R koyun\\ \tt \\ \tt Sıradaki komutunuz nedir?\\ \tt Komut: Dördüncü satırın üçüncü sütununa W koyun\\ \tt \\ \tt Sıradaki komutunuz nedir?\\ \tt Komut: BİTTİ 
	  } 
	   } 
	   } 
	 & & \\ 
 

    \theutterance \stepcounter{utterance}  

    & & & \multicolumn{2}{p{0.3\linewidth}}{\cellcolor[rgb]{0.95,0.95,0.95}{%
	\makecell[{{p{\linewidth}}}]{% 
	  \tt {\tiny [GM$|$GM]}  
	 Komut: İlk satırın ikinci sütununa R koyun 
	  } 
	   } 
	   } 
	 & & \\ 
 

    \theutterance \stepcounter{utterance}  

    & & \multicolumn{4}{p{0.6\linewidth}}{\cellcolor[rgb]{0.9,0.9,0.9}{%
	\makecell[{{p{\linewidth}}}]{% 
	  \tt {\tiny [GM$\rangle$B]}  
	 Birlikte bir şeyler çizelim. 5'e 5 boyutunda boş bir ızgara var, şöyle:\\ \tt \\ \tt ▢ ▢ ▢ ▢ ▢\\ \tt ▢ ▢ ▢ ▢ ▢\\ \tt ▢ ▢ ▢ ▢ ▢\\ \tt ▢ ▢ ▢ ▢ ▢\\ \tt ▢ ▢ ▢ ▢ ▢\\ \tt \\ \tt Size "sol üste bir X koy" gibi komutlar vereceğim ve siz de verilen komutu uygulayarak ızgarayı şu şekilde döndüreceksiniz:\\ \tt \\ \tt Komut: sol üst köşeye bir X koyun\\ \tt \\ \tt X ▢ ▢ ▢ ▢\\ \tt ▢ ▢ ▢ ▢ ▢\\ \tt ▢ ▢ ▢ ▢ ▢\\ \tt ▢ ▢ ▢ ▢ ▢\\ \tt ▢ ▢ ▢ ▢ ▢\\ \tt \\ \tt Şimdi 5'e 5 boyutunda boş bir ızgara oluşturun ve her adımda aşağıdaki komutları uygulayın. Komutu yürüttükten sonra yalnızca ızgarayı döndürün ve diğer tüm metni çıktıdan hariç tutun.\\ \tt Komut: İlk satırın ikinci sütununa R koyun 
	  } 
	   } 
	   } 
	 & \\ 
 

    \theutterance \stepcounter{utterance}  

    & & & \multicolumn{4}{p{0.6\linewidth}}{\cellcolor[rgb]{1,0.85,0.72}{%
	\makecell[{{p{\linewidth}}}]{% 
	  \tt {\tiny [GM$\langle$B]}  
	 R ▢ ▢ ▢ ▢\\ \tt ▢ ▢ ▢ ▢ ▢\\ \tt ▢ ▢ ▢ ▢ ▢\\ \tt ▢ ▢ ▢ ▢ ▢\\ \tt ▢ ▢ ▢ ▢ ▢ 
	  } 
	   } 
	   } 
	  \\ 
 

    \theutterance \stepcounter{utterance}  

    & & & \multicolumn{2}{p{0.3\linewidth}}{\cellcolor[rgb]{0.95,0.95,0.95}{%
	\makecell[{{p{\linewidth}}}]{% 
	  \tt {\tiny [GM$|$GM]}  
	 R ▢ ▢ ▢ ▢\\ \tt ▢ ▢ ▢ ▢ ▢\\ \tt ▢ ▢ ▢ ▢ ▢\\ \tt ▢ ▢ ▢ ▢ ▢\\ \tt ▢ ▢ ▢ ▢ ▢ 
	  } 
	   } 
	   } 
	 & & \\ 
 

    \theutterance \stepcounter{utterance}  

    & & \multicolumn{4}{p{0.6\linewidth}}{\cellcolor[rgb]{0.9,0.9,0.9}{%
	\makecell[{{p{\linewidth}}}]{% 
	  \tt {\tiny [A$\langle$GM]}  
	 Sıradaki komutunuz nedir? 
	  } 
	   } 
	   } 
	 & \\ 
 

    \theutterance \stepcounter{utterance}  

    & \multicolumn{4}{p{0.6\linewidth}}{\cellcolor[rgb]{0.8,1,0.9}{%
	\makecell[{{p{\linewidth}}}]{% 
	  \tt {\tiny [A$\rangle$GM]}  
	 Özür dilerim, önceki yanıtım hatalıydı. Daha önce belirttiğiniz ızgara zaten dolu olduğu için yeni komutlar veremiyoruz. Sadece "BİTTİ" diye bitirebiliriz. 
	  } 
	   } 
	   } 
	 & & \\ 
 

    \theutterance \stepcounter{utterance}  

    & & & \multicolumn{2}{p{0.3\linewidth}}{\cellcolor[rgb]{0.95,0.95,0.95}{%
	\makecell[{{p{\linewidth}}}]{% 
	  \tt {\tiny [GM$|$GM]}  
	 Invalid instruction format 
	  } 
	   } 
	   } 
	 & & \\ 
 

\end{supertabular}
}

\end{document}
