
\documentclass{article}
\usepackage{colortbl}
\usepackage{makecell}
\usepackage{multirow}
\usepackage{supertabular}

\begin{document}

\newcounter{utterance}

\twocolumn

{ \footnotesize  \setcounter{utterance}{1}
\setlength{\tabcolsep}{0pt}
\begin{supertabular}{c@{$\;$}|p{.15\linewidth}@{}p{.15\linewidth}p{.15\linewidth}p{.15\linewidth}p{.15\linewidth}p{.15\linewidth}}

    \# & $\;$A & \multicolumn{4}{c}{Game Master} & $\;\:$B\\
    \hline 

    \theutterance \stepcounter{utterance}  

    & & \multicolumn{4}{p{0.6\linewidth}}{\cellcolor[rgb]{0.9,0.9,0.9}{%
	\makecell[{{p{\linewidth}}}]{% 
	  \tt {\tiny [A$\langle$GM]}  
	 Lass uns die folgende Aufgabe lösen. Das Ziel ist, ein leeres Gitter zu füllen, das so aussieht:\\ \tt \\ \tt ▢ ▢ ▢ ▢ ▢\\ \tt ▢ ▢ ▢ ▢ ▢\\ \tt ▢ ▢ ▢ ▢ ▢\\ \tt ▢ ▢ ▢ ▢ ▢\\ \tt ▢ ▢ ▢ ▢ ▢\\ \tt \\ \tt Ein gefülltes Gitter der Größe 5x5 kann so aussehen:\\ \tt \\ \tt ▢ ▢ ▢ ▢ ▢\\ \tt ▢ ▢ E ▢ ▢\\ \tt ▢ ▢ ▢ ▢ ▢\\ \tt ▢ ▢ ▢ ▢ ▢\\ \tt X X X X X\\ \tt \\ \tt Ich möchte, dass du das Gitter Schritt für Schritt für mich beschreibst. Die leeren Zellen (durch „▢“ gekennzeichnet) musst du nicht beschreiben. Beschreibe nur die Position der Buchstaben im Gitter. Dann warte bis ich sage: „Was ist dein nächster Befehl?“, und mache dann mit dem nächsten Schritt weiter. Beziehe die Größe des Gitters in deine Befehle ein. Wenn du alles beschrieben hast, sage „FERTIG“.\\ \tt \\ \tt Hier sind Beispiel-Schritte für das gefüllte Gitter von oben.\\ \tt \\ \tt Was ist dein nächster Befehl?\\ \tt Befehl: Füge ein E in die zweite Reihe, dritte Spalte.\\ \tt \\ \tt Was ist dein nächster Befehl?\\ \tt Befehl: Fülle die letzte Reihe mit X.\\ \tt \\ \tt Was ist dein nächster Befehl?\\ \tt Befehl: FERTIG\\ \tt \\ \tt Ok. Nun führe dies für das folgende Beispiel aus, welches ein 5x5-Gitter ist.\\ \tt ▢ ▢ G ▢ ▢\\ \tt ▢ G ▢ ▢ G\\ \tt ▢ G ▢ ▢ ▢\\ \tt G ▢ G G ▢\\ \tt ▢ ▢ G ▢ ▢\\ \tt Was ist dein nächster Befehl?\\ \tt  
	  } 
	   } 
	   } 
	 & \\ 
 

    \theutterance \stepcounter{utterance}  

    & \multicolumn{4}{p{0.6\linewidth}}{\cellcolor[rgb]{0.8,1,0.9}{%
	\makecell[{{p{\linewidth}}}]{% 
	  \tt {\tiny [A$\rangle$GM]}  
	 Befehl: Füge ein G in die erste Reihe, zweite Spalte, ein.\\ \tt \\ \tt Was ist dein nächster Befehl? 
	  } 
	   } 
	   } 
	 & & \\ 
 

    \theutterance \stepcounter{utterance}  

    & & & \multicolumn{2}{p{0.3\linewidth}}{\cellcolor[rgb]{0.95,0.95,0.95}{%
	\makecell[{{p{\linewidth}}}]{% 
	  \tt {\tiny [GM$|$GM]}  
	 Invalid instruction format 
	  } 
	   } 
	   } 
	 & & \\ 
 

\end{supertabular}
}

\end{document}
