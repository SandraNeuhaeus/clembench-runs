
\documentclass{article}
\usepackage{colortbl}
\usepackage{makecell}
\usepackage{multirow}
\usepackage{supertabular}

\begin{document}

\newcounter{utterance}

\twocolumn

{ \footnotesize  \setcounter{utterance}{1}
\setlength{\tabcolsep}{0pt}
\begin{supertabular}{c@{$\;$}|p{.15\linewidth}@{}p{.15\linewidth}p{.15\linewidth}p{.15\linewidth}p{.15\linewidth}p{.15\linewidth}}

    \# & $\;$A & \multicolumn{4}{c}{Game Master} & $\;\:$B\\
    \hline

    \theutterance \stepcounter{utterance}

    & & \multicolumn{4}{p{0.6\linewidth}}{\cellcolor[rgb]{0.9,0.9,0.9}{%
	\makecell[{{p{\linewidth}}}]{% 
	  \tt {\tiny [A$\langle$GM]}  
	 మీకు మూడు గ్రిడ్‌లు ఇవ్వబడ్డాయి, వాటిలో ప్రతి ఒక్కటి 5 బై 5 పరిమాణంలో ఉంటాయి.\\ \tt గ్రిడ్‌లు "▢"తో గుర్తించబడిన ఖాళీ సెల్‌లను కలిగి ఉంటాయి మరియు "X"తో నిండిన సెల్‌లను కలిగి ఉంటాయి.\\ \tt లక్ష్య గ్రిడ్‌ని రెండు ఇతర డిస్ట్రాక్టర్ గ్రిడ్‌ల నుండి వేరు చేస్తూ దానిని ఉత్తమంగా వివరించే రెఫరింగ్ వ్యక్తీకరణను రూపొందించడం మీ పని.\\ \tt మొదటి గ్రిడ్ లక్ష్య గ్రిడ్, మరియు క్రింది రెండు గ్రిడ్‌లు డిస్ట్రాక్టర్‌లు.\\ \tt \\ \tt టార్గెట్ గ్రిడ్:\\ \tt \\ \tt X X X X X\\ \tt ▢ ▢ ▢ ▢ ▢\\ \tt X X X X X\\ \tt X X X X X\\ \tt X X X X X\\ \tt \\ \tt డిస్ట్రాక్టర్ గ్రిడ్ 1:\\ \tt \\ \tt ▢ ▢ ▢ ▢ ▢\\ \tt X X X X X\\ \tt X X X X X\\ \tt X X X X X\\ \tt X X X X X\\ \tt \\ \tt డిస్ట్రాక్టర్ గ్రిడ్ 2:\\ \tt \\ \tt X X X X X\\ \tt X X X X X\\ \tt ▢ ▢ ▢ ▢ ▢\\ \tt X X X X X\\ \tt X X X X X\\ \tt \\ \tt సూచన: లక్ష్య గ్రిడ్‌ను వివరించండి.\\ \tt ఇచ్చిన లక్ష్య గ్రిడ్ కోసం "వ్యక్తీకరణ:" ట్యాగ్‌తో ప్రారంభించి సూచించే వ్యక్తీకరణను రూపొందించండి. ఏదైనా ఇతర వచనాన్ని వదిలివేయండి. 
	  } 
	   } 
	   } 
	 & \\ 


    \theutterance \stepcounter{utterance}

    & \multicolumn{4}{p{0.6\linewidth}}{\cellcolor[rgb]{0.8,1,0.9}{%
	\makecell[{{p{\linewidth}}}]{% 
	  \tt {\tiny [A$\rangle$GM]}  
	 వ్యక్తీకరణ:\\ \tt \\ \tt లక్ష్య గ్రిడ్ ఒక 5x5 సమీప గ్రిడ్ అంటే, ప్రతి రెండు రెండు సె 
	  } 
	   } 
	   } 
	 & & \\ 


    \theutterance \stepcounter{utterance}

    & & & \multicolumn{2}{p{0.3\linewidth}}{\cellcolor[rgb]{0.95,0.95,0.95}{%
	\makecell[{{p{\linewidth}}}]{% 
	  \tt {\tiny [GM$|$GM]}  
	 వ్యక్తీకరణ:\\ \tt \\ \tt లక్ష్య గ్రిడ్ ఒక 5x5 సమీప గ్రిడ్ అంటే, ప్రతి రెండు రెండు సె 
	  } 
	   } 
	   } 
	 & & \\ 


    \theutterance \stepcounter{utterance}

    & & \multicolumn{4}{p{0.6\linewidth}}{\cellcolor[rgb]{0.9,0.9,0.9}{%
	\makecell[{{p{\linewidth}}}]{% 
	  \tt {\tiny [GM$\rangle$B]}  
	 మీకు మూడు గ్రిడ్‌లు ఇవ్వబడ్డాయి, వాటిలో ప్రతి ఒక్కటి 5 బై 5 పరిమాణంలో ఉంటాయి.\\ \tt గ్రిడ్‌లు "▢"తో గుర్తించబడిన ఖాళీ సెల్‌లను కలిగి ఉంటాయి మరియు "X"తో నిండిన సెల్‌లను కలిగి ఉంటాయి.\\ \tt ఇచ్చిన గ్రిడ్‌లలో ఒకదానిని వివరించే రెఫరింగ్ ఎక్స్‌ప్రెషన్ కూడా మీకు అందించబడింది.\\ \tt ఇచ్చిన రెఫరింగ్ ఎక్స్‌ప్రెషన్‌కు సరిపోయే గ్రిడ్‌ను ఎంచుకోవడం మీ పని.\\ \tt ప్రధమ, రెండవ లేదా మూడవది ఎంపికను ఎంచుకోవడం ద్వారా ఇచ్చిన వ్యక్తీకరణకు సరిపోలే గ్రిడ్ యొక్క సంఖ్యను (టెక్స్ట్‌లో) మాత్రమే రూపొందించండి.\\ \tt \\ \tt ప్రధమ:\\ \tt \\ \tt ▢ ▢ ▢ ▢ ▢\\ \tt X X X X X\\ \tt X X X X X\\ \tt X X X X X\\ \tt X X X X X\\ \tt \\ \tt రెండవ:\\ \tt \\ \tt X X X X X\\ \tt ▢ ▢ ▢ ▢ ▢\\ \tt X X X X X\\ \tt X X X X X\\ \tt X X X X X\\ \tt \\ \tt మూడవది:\\ \tt \\ \tt X X X X X\\ \tt X X X X X\\ \tt ▢ ▢ ▢ ▢ ▢\\ \tt X X X X X\\ \tt X X X X X\\ \tt \\ \tt వ్యక్తీకరణ:\\ \tt \\ \tt లక్ష్య గ్రిడ్ ఒక 5x5 సమీప గ్రిడ్ అంటే, ప్రతి రెండు రెండు సె\\ \tt ప్రశ్న: వ్యక్తీకరణ ఏ గ్రిడ్‌ను సూచిస్తుంది?\\ \tt మీ ఎంపిక తర్వాత "సమాధానం:" ట్యాగ్‌తో ప్రారంభించండి. ఏదైనా ఇతర వచనాన్ని వదిలివేయండి.\\ \tt  
	  } 
	   } 
	   } 
	 & \\ 


    \theutterance \stepcounter{utterance}

    & & & \multicolumn{4}{p{0.6\linewidth}}{\cellcolor[rgb]{1,0.85,0.72}{%
	\makecell[{{p{\linewidth}}}]{% 
	  \tt {\tiny [GM$\langle$B]}  
	 వ్యక్తీకరణ ఏ గ్రిడ్‌ను సూచిస్తుంది: ప్రధమ 
	  } 
	   } 
	   } 
	  \\ 


    \theutterance \stepcounter{utterance}

    & & & \multicolumn{2}{p{0.3\linewidth}}{\cellcolor[rgb]{0.95,0.95,0.95}{%
	\makecell[{{p{\linewidth}}}]{% 
	  \tt {\tiny [GM$|$GM]}  
	 Invalid generated choice 
	  } 
	   } 
	   } 
	 & & \\ 


\end{supertabular}
}

\end{document}
