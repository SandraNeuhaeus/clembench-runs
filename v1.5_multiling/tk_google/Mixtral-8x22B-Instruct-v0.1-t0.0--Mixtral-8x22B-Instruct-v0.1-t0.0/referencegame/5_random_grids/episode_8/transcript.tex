
\documentclass{article}
\usepackage{colortbl}
\usepackage{makecell}
\usepackage{multirow}
\usepackage{supertabular}

\begin{document}

\newcounter{utterance}

\twocolumn

{ \footnotesize  \setcounter{utterance}{1}
\setlength{\tabcolsep}{0pt}
\begin{supertabular}{c@{$\;$}|p{.15\linewidth}@{}p{.15\linewidth}p{.15\linewidth}p{.15\linewidth}p{.15\linewidth}p{.15\linewidth}}

    \# & $\;$A & \multicolumn{4}{c}{Game Master} & $\;\:$B\\
    \hline 

    \theutterance \stepcounter{utterance}  

    & & \multicolumn{4}{p{0.6\linewidth}}{\cellcolor[rgb]{0.9,0.9,0.9}{%
	\makecell[{{p{\linewidth}}}]{% 
	  \tt {\tiny [A$\langle$GM]}  
	 Size üç sany gözenek berilýär, olaryň hersiniň ululygy 5-den 5-e çenli.\\ \tt Torlarda "▢" bilen bellenen boş öýjükler we "X" bilen bellenen öýjükler bar.\\ \tt Siziň wezipäňiz, beýleki iki daşlaşdyryjy gözenekden tapawutlandyryp, maksat toruny has gowy suratlandyrýan salgylanma aňlatmasyny döretmekdir.\\ \tt Birinji gözenek maksatly gözenekdir we aşakdaky iki gözenek ünsüni bölýär.\\ \tt \\ \tt Maksatlaýyn set:\\ \tt \\ \tt X X X X ▢\\ \tt ▢ ▢ ▢ ▢ ▢\\ \tt X ▢ X ▢ X\\ \tt ▢ ▢ ▢ ▢ X\\ \tt ▢ ▢ ▢ X X\\ \tt \\ \tt Distraktor paneli 1:\\ \tt \\ \tt ▢ X X X ▢\\ \tt ▢ ▢ ▢ X ▢\\ \tt X ▢ X X ▢\\ \tt ▢ X ▢ ▢ ▢\\ \tt ▢ ▢ ▢ X X\\ \tt \\ \tt Distraktor paneli 2:\\ \tt \\ \tt X ▢ X ▢ ▢\\ \tt ▢ X ▢ ▢ ▢\\ \tt X ▢ X ▢ X\\ \tt ▢ ▢ X X ▢\\ \tt ▢ ▢ X X ▢\\ \tt \\ \tt Görkezme: Maksatlaýyn seti suratlandyryň.\\ \tt Berlen maksat paneli üçin "Aňlatmasyny:" belliginden başlap, salgylanma aňlatmasyny dörediň. Başga teksti goýuň. 
	  } 
	   } 
	   } 
	 & \\ 
 

    \theutterance \stepcounter{utterance}  

    & \multicolumn{4}{p{0.6\linewidth}}{\cellcolor[rgb]{0.8,1,0.9}{%
	\makecell[{{p{\linewidth}}}]{% 
	  \tt {\tiny [A$\rangle$GM]}  
	 I'm sorry for the inconvenience, but I'm currently only able to understand and generate text in English. However, I can still help you understand the problem and provide a solution.\\ \tt \\ \tt The problem seems to be a pattern recognition task, where you are given a target set and two distractor sets. The goal is to describe the pattern in the target set, which is distinguished from the distractor sets by the presence of two intersecting lines.\\ \tt \\ \tt Target 
	  } 
	   } 
	   } 
	 & & \\ 
 

    \theutterance \stepcounter{utterance}  

    & & & \multicolumn{2}{p{0.3\linewidth}}{\cellcolor[rgb]{0.95,0.95,0.95}{%
	\makecell[{{p{\linewidth}}}]{% 
	  \tt {\tiny [GM$|$GM]}  
	 Invalid generated expression 
	  } 
	   } 
	   } 
	 & & \\ 
 

\end{supertabular}
}

\end{document}
