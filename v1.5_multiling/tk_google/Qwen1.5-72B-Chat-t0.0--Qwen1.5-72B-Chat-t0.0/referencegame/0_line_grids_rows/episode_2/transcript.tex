
\documentclass{article}
\usepackage{colortbl}
\usepackage{makecell}
\usepackage{multirow}
\usepackage{supertabular}

\begin{document}

\newcounter{utterance}

\twocolumn

{ \footnotesize  \setcounter{utterance}{1}
\setlength{\tabcolsep}{0pt}
\begin{supertabular}{c@{$\;$}|p{.15\linewidth}@{}p{.15\linewidth}p{.15\linewidth}p{.15\linewidth}p{.15\linewidth}p{.15\linewidth}}

    \# & $\;$A & \multicolumn{4}{c}{Game Master} & $\;\:$B\\
    \hline 

    \theutterance \stepcounter{utterance}  

    & & \multicolumn{4}{p{0.6\linewidth}}{\cellcolor[rgb]{0.9,0.9,0.9}{%
	\makecell[{{p{\linewidth}}}]{% 
	  \tt {\tiny [A$\langle$GM]}  
	 Size üç sany gözenek berilýär, olaryň hersiniň ululygy 5-den 5-e çenli.\\ \tt Torlarda "▢" bilen bellenen boş öýjükler we "X" bilen bellenen öýjükler bar.\\ \tt Siziň wezipäňiz, beýleki iki daşlaşdyryjy gözenekden tapawutlandyryp, maksat toruny has gowy suratlandyrýan salgylanma aňlatmasyny döretmekdir.\\ \tt Birinji gözenek maksatly gözenekdir we aşakdaky iki gözenek ünsüni bölýär.\\ \tt \\ \tt Maksatlaýyn set:\\ \tt \\ \tt X X X X X\\ \tt ▢ ▢ ▢ ▢ ▢\\ \tt ▢ ▢ ▢ ▢ ▢\\ \tt ▢ ▢ ▢ ▢ ▢\\ \tt ▢ ▢ ▢ ▢ ▢\\ \tt \\ \tt Distraktor paneli 1:\\ \tt \\ \tt ▢ ▢ ▢ ▢ ▢\\ \tt X X X X X\\ \tt ▢ ▢ ▢ ▢ ▢\\ \tt ▢ ▢ ▢ ▢ ▢\\ \tt ▢ ▢ ▢ ▢ ▢\\ \tt \\ \tt Distraktor paneli 2:\\ \tt \\ \tt ▢ ▢ ▢ ▢ ▢\\ \tt ▢ ▢ ▢ ▢ ▢\\ \tt X X X X X\\ \tt ▢ ▢ ▢ ▢ ▢\\ \tt ▢ ▢ ▢ ▢ ▢\\ \tt \\ \tt Görkezme: Maksatlaýyn seti suratlandyryň.\\ \tt Berlen maksat paneli üçin "Aňlatmasyny:" belliginden başlap, salgylanma aňlatmasyny dörediň. Başga teksti goýuň. 
	  } 
	   } 
	   } 
	 & \\ 
 

    \theutterance \stepcounter{utterance}  

    & \multicolumn{4}{p{0.6\linewidth}}{\cellcolor[rgb]{0.8,1,0.9}{%
	\makecell[{{p{\linewidth}}}]{% 
	  \tt {\tiny [A$\rangle$GM]}  
	 Aňlatmasyny: \\ \tt \\ \tt Maksatlaýyn seti, beýleki iki daşlaşdyryjy gözenekten ibaret olup, ulanyjynyň dişlerine ulanylan bir dentastika protokollu aks etmektedir. Bu tor, ulanyjynyň dişleriň ululygyndan asyl, 5-dan 5-e çenlik halykara has 
	  } 
	   } 
	   } 
	 & & \\ 
 

    \theutterance \stepcounter{utterance}  

    & & & \multicolumn{2}{p{0.3\linewidth}}{\cellcolor[rgb]{0.95,0.95,0.95}{%
	\makecell[{{p{\linewidth}}}]{% 
	  \tt {\tiny [GM$|$GM]}  
	 Aňlatmasyny: \\ \tt \\ \tt Maksatlaýyn seti, beýleki iki daşlaşdyryjy gözenekten ibaret olup, ulanyjynyň dişlerine ulanylan bir dentastika protokollu aks etmektedir. Bu tor, ulanyjynyň dişleriň ululygyndan asyl, 5-dan 5-e çenlik halykara has 
	  } 
	   } 
	   } 
	 & & \\ 
 

    \theutterance \stepcounter{utterance}  

    & & \multicolumn{4}{p{0.6\linewidth}}{\cellcolor[rgb]{0.9,0.9,0.9}{%
	\makecell[{{p{\linewidth}}}]{% 
	  \tt {\tiny [GM$\rangle$B]}  
	 Size üç sany gözenek berilýär, olaryň hersiniň ululygy 5-den 5-e çenli.\\ \tt Torlarda "▢" bilen bellenen boş öýjükler we "X" bilen bellenen öýjükler bar.\\ \tt Şeýle hem size berlen gözenekleriň birini suratlandyrýan salgylanma berilýär.\\ \tt Siziň wezipäňiz, berlen salgylanma gabat gelýän seti saýlamakdyr.\\ \tt Diňe birinjisi, ikinjisi ýa-da üçünjisi birini saýlap, berlen aňlatma gabat gelýän gözenegiň sanyny (tekstde) dörediň.\\ \tt \\ \tt Birinjisi:\\ \tt \\ \tt ▢ ▢ ▢ ▢ ▢\\ \tt ▢ ▢ ▢ ▢ ▢\\ \tt X X X X X\\ \tt ▢ ▢ ▢ ▢ ▢\\ \tt ▢ ▢ ▢ ▢ ▢\\ \tt \\ \tt Ikinjisi:\\ \tt \\ \tt ▢ ▢ ▢ ▢ ▢\\ \tt X X X X X\\ \tt ▢ ▢ ▢ ▢ ▢\\ \tt ▢ ▢ ▢ ▢ ▢\\ \tt ▢ ▢ ▢ ▢ ▢\\ \tt \\ \tt Üçünjisi:\\ \tt \\ \tt X X X X X\\ \tt ▢ ▢ ▢ ▢ ▢\\ \tt ▢ ▢ ▢ ▢ ▢\\ \tt ▢ ▢ ▢ ▢ ▢\\ \tt ▢ ▢ ▢ ▢ ▢\\ \tt \\ \tt Aňlatmasyny: \\ \tt \\ \tt Maksatlaýyn seti, beýleki iki daşlaşdyryjy gözenekten ibaret olup, ulanyjynyň dişlerine ulanylan bir dentastika protokollu aks etmektedir. Bu tor, ulanyjynyň dişleriň ululygyndan asyl, 5-dan 5-e çenlik halykara has\\ \tt Sorag: Aňlatma haýsy seti aňladýar?\\ \tt "Jogap:" belligi bilen başlaň, soň bolsa saýlamagyňyzdan soň. Başga teksti goýuň.\\ \tt  
	  } 
	   } 
	   } 
	 & \\ 
 

    \theutterance \stepcounter{utterance}  

    & & & \multicolumn{4}{p{0.6\linewidth}}{\cellcolor[rgb]{1,0.85,0.72}{%
	\makecell[{{p{\linewidth}}}]{% 
	  \tt {\tiny [GM$\langle$B]}  
	 Aňlatma ikinjisi seti aňladýar. Bu tor, beýleki iki daşlaşdyryjy gözenekten ibaret olup, ulanyjynyň dişleriň ululygyndan asyl, 5-dan 5-e çenlik halykara has etmedikçe, maksatla dişlerin normal ululygyna qaytarılması we dişligi sa 
	  } 
	   } 
	   } 
	  \\ 
 

    \theutterance \stepcounter{utterance}  

    & & & \multicolumn{2}{p{0.3\linewidth}}{\cellcolor[rgb]{0.95,0.95,0.95}{%
	\makecell[{{p{\linewidth}}}]{% 
	  \tt {\tiny [GM$|$GM]}  
	 Invalid generated choice 
	  } 
	   } 
	   } 
	 & & \\ 
 

\end{supertabular}
}

\end{document}
