
\documentclass{article}
\usepackage{colortbl}
\usepackage{makecell}
\usepackage{multirow}
\usepackage{supertabular}

\begin{document}

\newcounter{utterance}

\twocolumn

{ \footnotesize  \setcounter{utterance}{1}
\setlength{\tabcolsep}{0pt}
\begin{supertabular}{c@{$\;$}|p{.15\linewidth}@{}p{.15\linewidth}p{.15\linewidth}p{.15\linewidth}p{.15\linewidth}p{.15\linewidth}}

    \# & $\;$A & \multicolumn{4}{c}{Game Master} & $\;\:$B\\
    \hline

    \theutterance \stepcounter{utterance}

    & & \multicolumn{4}{p{0.6\linewidth}}{\cellcolor[rgb]{0.9,0.9,0.9}{%
	\makecell[{{p{\linewidth}}}]{% 
	  \tt {\tiny [A$\langle$GM]}  
	 Вам даны три сетки, каждая из которых имеет размер 5 на 5.\\ \tt В сетках есть пустые ячейки, отмеченные знаком «▢», и заполненные ячейки, отмеченные знаком «X».\\ \tt Ваша задача — создать ссылающееся выражение, которое лучше всего описывает целевую сетку, отличая ее от двух других сеток-отвлекателей.\\ \tt Первая сетка — это целевая сетка, а следующие две сетки — дистракторы.\\ \tt \\ \tt Целевая сетка:\\ \tt \\ \tt ▢ ▢ ▢ ▢ X\\ \tt X ▢ X ▢ ▢\\ \tt ▢ ▢ X ▢ X\\ \tt X ▢ X ▢ ▢\\ \tt X X ▢ ▢ X\\ \tt \\ \tt Отвлекающая сетка 1:\\ \tt \\ \tt X X ▢ ▢ X\\ \tt X ▢ X ▢ ▢\\ \tt X ▢ ▢ X ▢\\ \tt ▢ ▢ X ▢ ▢\\ \tt ▢ X ▢ X ▢\\ \tt \\ \tt Отвлекающая сетка 2:\\ \tt \\ \tt ▢ ▢ ▢ ▢ ▢\\ \tt ▢ ▢ ▢ ▢ ▢\\ \tt ▢ X ▢ X ▢\\ \tt X X X ▢ X\\ \tt X X X ▢ X\\ \tt \\ \tt Инструкция: Опишите целевую сетку.\\ \tt Создайте ссылающееся выражение, начинающееся с тега «Выражение:» для данной целевой сетки. Опустите любой другой текст. 
	  } 
	   } 
	   } 
	 & \\ 


    \theutterance \stepcounter{utterance}

    & \multicolumn{4}{p{0.6\linewidth}}{\cellcolor[rgb]{0.8,1,0.9}{%
	\makecell[{{p{\linewidth}}}]{% 
	  \tt {\tiny [A$\rangle$GM]}  
	 Expression: The target grid is characterized by a pattern where the first column has two "X" marks, the second column has one "X" mark, the third column has two "X" marks, the fourth column has one "X" mark, and the fifth column has two "X" marks. Additionally, the "X" marks in each row are positioned such that there is at least one empty cell between any two "X" marks. 
	  } 
	   } 
	   } 
	 & & \\ 


    \theutterance \stepcounter{utterance}

    & & & \multicolumn{2}{p{0.3\linewidth}}{\cellcolor[rgb]{0.95,0.95,0.95}{%
	\makecell[{{p{\linewidth}}}]{% 
	  \tt {\tiny [GM$|$GM]}  
	 Invalid generated expression 
	  } 
	   } 
	   } 
	 & & \\ 


\end{supertabular}
}

\end{document}
