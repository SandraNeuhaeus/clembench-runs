
\documentclass{article}
\usepackage{colortbl}
\usepackage{makecell}
\usepackage{multirow}
\usepackage{supertabular}

\begin{document}

\newcounter{utterance}

\twocolumn

{ \footnotesize  \setcounter{utterance}{1}
\setlength{\tabcolsep}{0pt}
\begin{supertabular}{c@{$\;$}|p{.15\linewidth}@{}p{.15\linewidth}p{.15\linewidth}p{.15\linewidth}p{.15\linewidth}p{.15\linewidth}}

    \# & $\;$A & \multicolumn{4}{c}{Game Master} & $\;\:$B\\
    \hline 

    \theutterance \stepcounter{utterance}  

    & & \multicolumn{4}{p{0.6\linewidth}}{\cellcolor[rgb]{0.9,0.9,0.9}{%
	\makecell[{{p{\linewidth}}}]{% 
	  \tt {\tiny [A$\langle$GM]}  
	 Вам даны три сетки, каждая из которых имеет размер 5 на 5.\\ \tt В сетках есть пустые ячейки, отмеченные знаком «▢», и заполненные ячейки, отмеченные знаком «X».\\ \tt Ваша задача — создать ссылающееся выражение, которое лучше всего описывает целевую сетку, отличая ее от двух других сеток-отвлекателей.\\ \tt Первая сетка — это целевая сетка, а следующие две сетки — дистракторы.\\ \tt \\ \tt Целевая сетка:\\ \tt \\ \tt ▢ ▢ ▢ ▢ ▢\\ \tt ▢ ▢ ▢ ▢ ▢\\ \tt ▢ ▢ ▢ ▢ ▢\\ \tt X X X X X\\ \tt ▢ ▢ ▢ ▢ ▢\\ \tt \\ \tt Отвлекающая сетка 1:\\ \tt \\ \tt X X X X X\\ \tt ▢ ▢ ▢ ▢ ▢\\ \tt ▢ ▢ ▢ ▢ ▢\\ \tt ▢ ▢ ▢ ▢ ▢\\ \tt ▢ ▢ ▢ ▢ ▢\\ \tt \\ \tt Отвлекающая сетка 2:\\ \tt \\ \tt ▢ ▢ ▢ ▢ ▢\\ \tt X X X X X\\ \tt ▢ ▢ ▢ ▢ ▢\\ \tt ▢ ▢ ▢ ▢ ▢\\ \tt ▢ ▢ ▢ ▢ ▢\\ \tt \\ \tt Инструкция: Опишите целевую сетку.\\ \tt Создайте ссылающееся выражение, начинающееся с тега «Выражение:» для данной целевой сетки. Опустите любой другой текст. 
	  } 
	   } 
	   } 
	 & \\ 
 

    \theutterance \stepcounter{utterance}  

    & \multicolumn{4}{p{0.6\linewidth}}{\cellcolor[rgb]{0.8,1,0.9}{%
	\makecell[{{p{\linewidth}}}]{% 
	  \tt {\tiny [A$\rangle$GM]}  
	 Выражение: В нижнем ряду все пять ячеек заполнены символами "X". 
	  } 
	   } 
	   } 
	 & & \\ 
 

    \theutterance \stepcounter{utterance}  

    & & & \multicolumn{2}{p{0.3\linewidth}}{\cellcolor[rgb]{0.95,0.95,0.95}{%
	\makecell[{{p{\linewidth}}}]{% 
	  \tt {\tiny [GM$|$GM]}  
	 Выражение: В нижнем ряду все пять ячеек заполнены символами "X". 
	  } 
	   } 
	   } 
	 & & \\ 
 

    \theutterance \stepcounter{utterance}  

    & & \multicolumn{4}{p{0.6\linewidth}}{\cellcolor[rgb]{0.9,0.9,0.9}{%
	\makecell[{{p{\linewidth}}}]{% 
	  \tt {\tiny [GM$\rangle$B]}  
	 Вам даны три сетки, каждая из которых имеет размер 5 на 5.\\ \tt В сетках есть пустые ячейки, отмеченные знаком «▢», и заполненные ячейки, отмеченные знаком «X».\\ \tt Вам также дано ссылающееся выражение, описывающее одну из заданных сеток.\\ \tt Ваша задача — выбрать сетку, соответствующую данному ссылающемуся выражению.\\ \tt Сгенерируйте только тот номер (в тексте) сетки, который соответствует данному выражению, выбрав первый, второй или третий.\\ \tt \\ \tt Первый:\\ \tt \\ \tt ▢ ▢ ▢ ▢ ▢\\ \tt ▢ ▢ ▢ ▢ ▢\\ \tt ▢ ▢ ▢ ▢ ▢\\ \tt X X X X X\\ \tt ▢ ▢ ▢ ▢ ▢\\ \tt \\ \tt Второй:\\ \tt \\ \tt X X X X X\\ \tt ▢ ▢ ▢ ▢ ▢\\ \tt ▢ ▢ ▢ ▢ ▢\\ \tt ▢ ▢ ▢ ▢ ▢\\ \tt ▢ ▢ ▢ ▢ ▢\\ \tt \\ \tt Третий:\\ \tt \\ \tt ▢ ▢ ▢ ▢ ▢\\ \tt X X X X X\\ \tt ▢ ▢ ▢ ▢ ▢\\ \tt ▢ ▢ ▢ ▢ ▢\\ \tt ▢ ▢ ▢ ▢ ▢\\ \tt \\ \tt Выражение: В нижнем ряду все пять ячеек заполнены символами "X".\\ \tt Вопрос: К какой сетке относится это выражение?\\ \tt Начните с тега «Ответ:», за которым следует ваш выбор. Опустите любой другой текст.\\ \tt  
	  } 
	   } 
	   } 
	 & \\ 
 

    \theutterance \stepcounter{utterance}  

    & & & \multicolumn{4}{p{0.6\linewidth}}{\cellcolor[rgb]{1,0.85,0.72}{%
	\makecell[{{p{\linewidth}}}]{% 
	  \tt {\tiny [GM$\langle$B]}  
	 Ответ: Второй 
	  } 
	   } 
	   } 
	  \\ 
 

    \theutterance \stepcounter{utterance}  

    & & & \multicolumn{2}{p{0.3\linewidth}}{\cellcolor[rgb]{0.95,0.95,0.95}{%
	\makecell[{{p{\linewidth}}}]{% 
	  \tt {\tiny [GM$|$GM]}  
	 Ответ: Второй 
	  } 
	   } 
	   } 
	 & & \\ 
 

\end{supertabular}
}

\end{document}
