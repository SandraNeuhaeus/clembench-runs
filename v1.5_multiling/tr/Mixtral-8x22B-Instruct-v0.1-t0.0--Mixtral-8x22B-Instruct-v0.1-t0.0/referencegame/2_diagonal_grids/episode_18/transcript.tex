
\documentclass{article}
\usepackage{colortbl}
\usepackage{makecell}
\usepackage{multirow}
\usepackage{supertabular}

\begin{document}

\newcounter{utterance}

\twocolumn

{ \footnotesize  \setcounter{utterance}{1}
\setlength{\tabcolsep}{0pt}
\begin{supertabular}{c@{$\;$}|p{.15\linewidth}@{}p{.15\linewidth}p{.15\linewidth}p{.15\linewidth}p{.15\linewidth}p{.15\linewidth}}

    \# & $\;$A & \multicolumn{4}{c}{Game Master} & $\;\:$B\\
    \hline 

    \theutterance \stepcounter{utterance}  

    & & \multicolumn{4}{p{0.6\linewidth}}{\cellcolor[rgb]{0.9,0.9,0.9}{%
	\makecell[{{p{\linewidth}}}]{% 
	  \tt {\tiny [A$\langle$GM]}  
	 Size her biri 5'e 5 boyutunda olan üç matris veriliyor.\\ \tt Matrislerde "▢" ile işaretlenmiş boş hücreler ve "X" ile işaretlenmiş dolu hücreler vardır.\\ \tt Göreviniz, hedef matrisi diğer iki çeldirici matristen ayıran ve onu en iyi şekilde tanımlayan bir referans ifadesi oluşturmaktır.\\ \tt İlk matris hedef matristir ve sonraki iki matris çeldiricilerdir.\\ \tt \\ \tt Hedef matris:\\ \tt \\ \tt X ▢ ▢ ▢ ▢\\ \tt X X ▢ ▢ ▢\\ \tt X X X ▢ ▢\\ \tt X X X X ▢\\ \tt X X X X X\\ \tt \\ \tt Çeldirici matris 1:\\ \tt \\ \tt X ▢ ▢ ▢ ▢\\ \tt ▢ X ▢ ▢ ▢\\ \tt ▢ ▢ X ▢ ▢\\ \tt ▢ ▢ ▢ X ▢\\ \tt ▢ ▢ ▢ ▢ X\\ \tt \\ \tt Çeldirici matris 2:\\ \tt \\ \tt X ▢ ▢ ▢ X\\ \tt ▢ X ▢ X ▢\\ \tt ▢ ▢ X ▢ ▢\\ \tt ▢ X ▢ X ▢\\ \tt X ▢ ▢ ▢ X\\ \tt \\ \tt Talimat: Hedef matrisi tanımlayın.\\ \tt "İfade: " etiketi ile başlayan referans ifadesini oluşturun. Diğer metinleri atlayın. 
	  } 
	   } 
	   } 
	 & \\ 
 

    \theutterance \stepcounter{utterance}  

    & \multicolumn{4}{p{0.6\linewidth}}{\cellcolor[rgb]{0.8,1,0.9}{%
	\makecell[{{p{\linewidth}}}]{% 
	  \tt {\tiny [A$\rangle$GM]}  
	 "Expression: The target matrix is characterized by a pattern of 'X' marks that form a diagonal line from the top left to the bottom right, with all cells above this diagonal being filled with 'X' marks and all cells below being empty." 
	  } 
	   } 
	   } 
	 & & \\ 
 

    \theutterance \stepcounter{utterance}  

    & & & \multicolumn{2}{p{0.3\linewidth}}{\cellcolor[rgb]{0.95,0.95,0.95}{%
	\makecell[{{p{\linewidth}}}]{% 
	  \tt {\tiny [GM$|$GM]}  
	 Invalid generated expression 
	  } 
	   } 
	   } 
	 & & \\ 
 

\end{supertabular}
}

\end{document}
