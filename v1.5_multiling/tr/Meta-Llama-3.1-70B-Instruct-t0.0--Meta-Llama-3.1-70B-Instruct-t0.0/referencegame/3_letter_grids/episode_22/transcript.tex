
\documentclass{article}
\usepackage{colortbl}
\usepackage{makecell}
\usepackage{multirow}
\usepackage{supertabular}

\begin{document}

\newcounter{utterance}

\twocolumn

{ \footnotesize  \setcounter{utterance}{1}
\setlength{\tabcolsep}{0pt}
\begin{supertabular}{c@{$\;$}|p{.15\linewidth}@{}p{.15\linewidth}p{.15\linewidth}p{.15\linewidth}p{.15\linewidth}p{.15\linewidth}}

    \# & $\;$A & \multicolumn{4}{c}{Game Master} & $\;\:$B\\
    \hline

    \theutterance \stepcounter{utterance}

    & & \multicolumn{4}{p{0.6\linewidth}}{\cellcolor[rgb]{0.9,0.9,0.9}{%
	\makecell[{{p{\linewidth}}}]{% 
	  \tt {\tiny [A$\langle$GM]}  
	 Size her biri 5'e 5 boyutunda olan üç matris veriliyor.\\ \tt Matrislerde "▢" ile işaretlenmiş boş hücreler ve "X" ile işaretlenmiş dolu hücreler vardır.\\ \tt Göreviniz, hedef matrisi diğer iki çeldirici matristen ayıran ve onu en iyi şekilde tanımlayan bir referans ifadesi oluşturmaktır.\\ \tt İlk matris hedef matristir ve sonraki iki matris çeldiricilerdir.\\ \tt \\ \tt Hedef matris:\\ \tt \\ \tt X X X X X\\ \tt ▢ ▢ X ▢ ▢\\ \tt ▢ ▢ X ▢ ▢\\ \tt ▢ ▢ X ▢ ▢\\ \tt ▢ ▢ X ▢ ▢\\ \tt \\ \tt Çeldirici matris 1:\\ \tt \\ \tt X ▢ ▢ ▢ X\\ \tt ▢ X ▢ X ▢\\ \tt ▢ ▢ X ▢ ▢\\ \tt ▢ ▢ X ▢ ▢\\ \tt ▢ ▢ X ▢ ▢\\ \tt \\ \tt Çeldirici matris 2:\\ \tt \\ \tt X X X X X\\ \tt X ▢ ▢ ▢ ▢\\ \tt X X X ▢ ▢\\ \tt X ▢ ▢ ▢ ▢\\ \tt X ▢ ▢ ▢ ▢\\ \tt \\ \tt Talimat: Hedef matrisi tanımlayın.\\ \tt "İfade: " etiketi ile başlayan referans ifadesini oluşturun. Diğer metinleri atlayın. 
	  } 
	   } 
	   } 
	 & \\ 


    \theutterance \stepcounter{utterance}

    & \multicolumn{4}{p{0.6\linewidth}}{\cellcolor[rgb]{0.8,1,0.9}{%
	\makecell[{{p{\linewidth}}}]{% 
	  \tt {\tiny [A$\rangle$GM]}  
	 İfade: Hedef matris, dikey olarak ortada yer alan "X" hücrelerinin bir sütun oluşturduğu ve bu sütunun her iki tarafında boş hücrelerin yer aldığı matristir. 
	  } 
	   } 
	   } 
	 & & \\ 


    \theutterance \stepcounter{utterance}

    & & & \multicolumn{2}{p{0.3\linewidth}}{\cellcolor[rgb]{0.95,0.95,0.95}{%
	\makecell[{{p{\linewidth}}}]{% 
	  \tt {\tiny [GM$|$GM]}  
	 İfade: Hedef matris, dikey olarak ortada yer alan "X" hücrelerinin bir sütun oluşturduğu ve bu sütunun her iki tarafında boş hücrelerin yer aldığı matristir. 
	  } 
	   } 
	   } 
	 & & \\ 


    \theutterance \stepcounter{utterance}

    & & \multicolumn{4}{p{0.6\linewidth}}{\cellcolor[rgb]{0.9,0.9,0.9}{%
	\makecell[{{p{\linewidth}}}]{% 
	  \tt {\tiny [GM$\rangle$B]}  
	 Size her biri 5'e 5 boyutunda olan üç matris verilir.\\ \tt Izgaralarda "▢" ile işaretlenmiş boş hücreler ve "X" ile işaretlenmiş dolu hücreler vardır.\\ \tt Ayrıca size verilen matrislerden birini tanımlayan bir referans ifadesi de veriliyor.\\ \tt Göreviniz, verilen referans ifadeyle eşleşen matrisi seçmektir.\\ \tt Verilen ifadeyle eşleşen matrisin yalnızca numarasını (metin olarak) birinci, ikinci veya üçüncü olarak oluşturun.\\ \tt \\ \tt Birinci:\\ \tt \\ \tt X ▢ ▢ ▢ X\\ \tt ▢ X ▢ X ▢\\ \tt ▢ ▢ X ▢ ▢\\ \tt ▢ ▢ X ▢ ▢\\ \tt ▢ ▢ X ▢ ▢\\ \tt \\ \tt İkinci:\\ \tt \\ \tt X X X X X\\ \tt ▢ ▢ X ▢ ▢\\ \tt ▢ ▢ X ▢ ▢\\ \tt ▢ ▢ X ▢ ▢\\ \tt ▢ ▢ X ▢ ▢\\ \tt \\ \tt Üçüncü:\\ \tt \\ \tt X X X X X\\ \tt X ▢ ▢ ▢ ▢\\ \tt X X X ▢ ▢\\ \tt X ▢ ▢ ▢ ▢\\ \tt X ▢ ▢ ▢ ▢\\ \tt \\ \tt İfade: Hedef matris, dikey olarak ortada yer alan "X" hücrelerinin bir sütun oluşturduğu ve bu sütunun her iki tarafında boş hücrelerin yer aldığı matristir.\\ \tt Soru: İfade hangi matrise denk geliyor?\\ \tt "Cevap: " etiketiyle başlayın ve ardından seçiminizi yazın. Diğer metinleri atlayın.\\ \tt  
	  } 
	   } 
	   } 
	 & \\ 


    \theutterance \stepcounter{utterance}

    & & & \multicolumn{4}{p{0.6\linewidth}}{\cellcolor[rgb]{1,0.85,0.72}{%
	\makecell[{{p{\linewidth}}}]{% 
	  \tt {\tiny [GM$\langle$B]}  
	 Cevap: İkinci 
	  } 
	   } 
	   } 
	  \\ 


    \theutterance \stepcounter{utterance}

    & & & \multicolumn{2}{p{0.3\linewidth}}{\cellcolor[rgb]{0.95,0.95,0.95}{%
	\makecell[{{p{\linewidth}}}]{% 
	  \tt {\tiny [GM$|$GM]}  
	 Cevap: İkinci 
	  } 
	   } 
	   } 
	 & & \\ 


\end{supertabular}
}

\end{document}
