
\documentclass{article}
\usepackage{colortbl}
\usepackage{makecell}
\usepackage{multirow}
\usepackage{supertabular}

\begin{document}

\newcounter{utterance}

\twocolumn

{ \footnotesize  \setcounter{utterance}{1}
\setlength{\tabcolsep}{0pt}
\begin{supertabular}{c@{$\;$}|p{.15\linewidth}@{}p{.15\linewidth}p{.15\linewidth}p{.15\linewidth}p{.15\linewidth}p{.15\linewidth}}

    \# & $\;$A & \multicolumn{4}{c}{Game Master} & $\;\:$B\\
    \hline 

    \theutterance \stepcounter{utterance}  

    & & \multicolumn{4}{p{0.6\linewidth}}{\cellcolor[rgb]{0.9,0.9,0.9}{%
	\makecell[{{p{\linewidth}}}]{% 
	  \tt {\tiny [A$\langle$GM]}  
	 Size üç sany gözenek berilýär, olaryň hersiniň ululygy 5-den 5-e çenli.\\ \tt Gözeneklerde "▢" bilen bellenen boş öýjükler we "X" bilen bellenen doly öýjükler bar.\\ \tt Siziň wezipäňiz, beýleki iki daşlaşdyryjy gözenekden tapawutlandyryp, maksat gözenegi has gowy teswirleÿän aňlatmany döretmekdir.\\ \tt Birinji gözenek maksatly gözenekdir we aşakdaky iki gözenek daşlaşdyryjydyr.\\ \tt \\ \tt Maksatly gözenek:\\ \tt \\ \tt X ▢ ▢ ▢ X\\ \tt ▢ X ▢ X ▢\\ \tt ▢ ▢ X ▢ ▢\\ \tt ▢ X ▢ X ▢\\ \tt X ▢ ▢ ▢ X\\ \tt \\ \tt Daşlaşdyryjy gözenek 1:\\ \tt \\ \tt ▢ ▢ ▢ ▢ X\\ \tt ▢ ▢ ▢ X ▢\\ \tt ▢ ▢ X ▢ ▢\\ \tt ▢ X ▢ ▢ ▢\\ \tt X ▢ ▢ ▢ ▢\\ \tt \\ \tt Daşlaşdyryjy gözenek 2:\\ \tt \\ \tt X ▢ ▢ ▢ ▢\\ \tt ▢ X ▢ ▢ ▢\\ \tt ▢ ▢ X ▢ ▢\\ \tt ▢ ▢ ▢ X ▢\\ \tt ▢ ▢ ▢ ▢ X\\ \tt \\ \tt Görkezme: Maksatly gözenegi teswirläň.\\ \tt Berlen maksat gözenegi üçin "Aňlatma:" belligi bilen başlap oňa gabat gelýän aňlatmasyny dörediň. Başga tekst döretme. 
	  } 
	   } 
	   } 
	 & \\ 
 

    \theutterance \stepcounter{utterance}  

    & \multicolumn{4}{p{0.6\linewidth}}{\cellcolor[rgb]{0.8,1,0.9}{%
	\makecell[{{p{\linewidth}}}]{% 
	  \tt {\tiny [A$\rangle$GM]}  
	 Based on the information provided, I understand that there are three grids, each with a size of 5x5. The grids contain empty spaces marked with "▢" and filled spaces marked with "X". Your task is to create an explanation that distinguishes the target grid from the two other grids. The first grid is the target grid, and the other two are the distracting grids.\\ \tt \\ \tt Target Grid:\\ \tt \\ \tt X ▢ 
	  } 
	   } 
	   } 
	 & & \\ 
 

    \theutterance \stepcounter{utterance}  

    & & & \multicolumn{2}{p{0.3\linewidth}}{\cellcolor[rgb]{0.95,0.95,0.95}{%
	\makecell[{{p{\linewidth}}}]{% 
	  \tt {\tiny [GM$|$GM]}  
	 Invalid generated expression 
	  } 
	   } 
	   } 
	 & & \\ 
 

\end{supertabular}
}

\end{document}
