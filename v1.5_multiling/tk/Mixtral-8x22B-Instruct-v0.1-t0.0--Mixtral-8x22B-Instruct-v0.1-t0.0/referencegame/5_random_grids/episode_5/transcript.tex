
\documentclass{article}
\usepackage{colortbl}
\usepackage{makecell}
\usepackage{multirow}
\usepackage{supertabular}

\begin{document}

\newcounter{utterance}

\twocolumn

{ \footnotesize  \setcounter{utterance}{1}
\setlength{\tabcolsep}{0pt}
\begin{supertabular}{c@{$\;$}|p{.15\linewidth}@{}p{.15\linewidth}p{.15\linewidth}p{.15\linewidth}p{.15\linewidth}p{.15\linewidth}}

    \# & $\;$A & \multicolumn{4}{c}{Game Master} & $\;\:$B\\
    \hline

    \theutterance \stepcounter{utterance}

    & & \multicolumn{4}{p{0.6\linewidth}}{\cellcolor[rgb]{0.9,0.9,0.9}{%
	\makecell[{{p{\linewidth}}}]{% 
	  \tt {\tiny [A$\langle$GM]}  
	 Size üç sany gözenek berilýär, olaryň hersiniň ululygy 5-den 5-e çenli.\\ \tt Gözeneklerde "▢" bilen bellenen boş öýjükler we "X" bilen bellenen doly öýjükler bar.\\ \tt Siziň wezipäňiz, beýleki iki daşlaşdyryjy gözenekden tapawutlandyryp, maksat gözenegi has gowy teswirleÿän aňlatmany döretmekdir.\\ \tt Birinji gözenek maksatly gözenekdir we aşakdaky iki gözenek daşlaşdyryjydyr.\\ \tt \\ \tt Maksatly gözenek:\\ \tt \\ \tt X X ▢ ▢ X\\ \tt X ▢ X ▢ ▢\\ \tt X ▢ ▢ X ▢\\ \tt ▢ ▢ X ▢ ▢\\ \tt ▢ X ▢ X ▢\\ \tt \\ \tt Daşlaşdyryjy gözenek 1:\\ \tt \\ \tt X ▢ X ▢ X\\ \tt ▢ ▢ ▢ X X\\ \tt ▢ ▢ ▢ X ▢\\ \tt X ▢ X ▢ ▢\\ \tt ▢ X X ▢ ▢\\ \tt \\ \tt Daşlaşdyryjy gözenek 2:\\ \tt \\ \tt X X ▢ ▢ ▢\\ \tt ▢ ▢ X ▢ X\\ \tt ▢ X X ▢ ▢\\ \tt ▢ X X ▢ ▢\\ \tt X ▢ ▢ X ▢\\ \tt \\ \tt Görkezme: Maksatly gözenegi teswirläň.\\ \tt Berlen maksat gözenegi üçin "Aňlatma:" belligi bilen başlap oňa gabat gelýän aňlatmasyny dörediň. Başga tekst döretme. 
	  } 
	   } 
	   } 
	 & \\ 


    \theutterance \stepcounter{utterance}

    & \multicolumn{4}{p{0.6\linewidth}}{\cellcolor[rgb]{0.8,1,0.9}{%
	\makecell[{{p{\linewidth}}}]{% 
	  \tt {\tiny [A$\rangle$GM]}  
	 I'm sorry for any confusion, but it seems you're asking for an explanation of a puzzle or a game in a language I'm not currently able to understand or translate. However, I can provide a general approach to solving such puzzles.\\ \tt \\ \tt The puzzle appears to be a type of logic puzzle, where you have a target pattern (the "maksatly gözenek") and two other patterns (the "daşlaşdyryjy gö 
	  } 
	   } 
	   } 
	 & & \\ 


    \theutterance \stepcounter{utterance}

    & & & \multicolumn{2}{p{0.3\linewidth}}{\cellcolor[rgb]{0.95,0.95,0.95}{%
	\makecell[{{p{\linewidth}}}]{% 
	  \tt {\tiny [GM$|$GM]}  
	 Invalid generated expression 
	  } 
	   } 
	   } 
	 & & \\ 


\end{supertabular}
}

\end{document}
