
\documentclass{article}
\usepackage{colortbl}
\usepackage{makecell}
\usepackage{multirow}
\usepackage{supertabular}

\begin{document}

\newcounter{utterance}

\twocolumn

{ \footnotesize  \setcounter{utterance}{1}
\setlength{\tabcolsep}{0pt}
\begin{supertabular}{c@{$\;$}|p{.15\linewidth}@{}p{.15\linewidth}p{.15\linewidth}p{.15\linewidth}p{.15\linewidth}p{.15\linewidth}}

    \# & $\;$A & \multicolumn{4}{c}{Game Master} & $\;\:$B\\
    \hline

    \theutterance \stepcounter{utterance}

    & & \multicolumn{4}{p{0.6\linewidth}}{\cellcolor[rgb]{0.9,0.9,0.9}{%
	\makecell[{{p{\linewidth}}}]{% 
	  \tt {\tiny [A$\langle$GM]}  
	 Size üç sany gözenek berilýär, olaryň hersiniň ululygy 5-den 5-e çenli.\\ \tt Gözeneklerde "▢" bilen bellenen boş öýjükler we "X" bilen bellenen doly öýjükler bar.\\ \tt Siziň wezipäňiz, beýleki iki daşlaşdyryjy gözenekden tapawutlandyryp, maksat gözenegi has gowy teswirleÿän aňlatmany döretmekdir.\\ \tt Birinji gözenek maksatly gözenekdir we aşakdaky iki gözenek daşlaşdyryjydyr.\\ \tt \\ \tt Maksatly gözenek:\\ \tt \\ \tt X ▢ ▢ ▢ X\\ \tt X ▢ ▢ ▢ X\\ \tt X X X X X\\ \tt X ▢ ▢ ▢ X\\ \tt X ▢ ▢ ▢ X\\ \tt \\ \tt Daşlaşdyryjy gözenek 1:\\ \tt \\ \tt X ▢ ▢ ▢ X\\ \tt X X ▢ X X\\ \tt X ▢ X ▢ X\\ \tt X ▢ ▢ ▢ X\\ \tt X ▢ ▢ ▢ X\\ \tt \\ \tt Daşlaşdyryjy gözenek 2:\\ \tt \\ \tt X ▢ ▢ ▢ X\\ \tt X X ▢ ▢ X\\ \tt X ▢ X ▢ X\\ \tt X ▢ ▢ X X\\ \tt X ▢ ▢ ▢ X\\ \tt \\ \tt Görkezme: Maksatly gözenegi teswirläň.\\ \tt Berlen maksat gözenegi üçin "Aňlatma:" belligi bilen başlap oňa gabat gelýän aňlatmasyny dörediň. Başga tekst döretme. 
	  } 
	   } 
	   } 
	 & \\ 


    \theutterance \stepcounter{utterance}

    & \multicolumn{4}{p{0.6\linewidth}}{\cellcolor[rgb]{0.8,1,0.9}{%
	\makecell[{{p{\linewidth}}}]{% 
	  \tt {\tiny [A$\rangle$GM]}  
	 Aňlatma:\\ \tt \\ \tt Maksat gözenegi, beýleki iki daşlaşdyryjy gözenekden tapawutlanýar. Ol, maksatly gözenekdi we olaryň ortasynda doly öýjükler ("X") bar. Bu gözenekde, maksatly öýjükler ("X") gözeneklerin ortasynda ýerleşip, olaryň ululygy 
	  } 
	   } 
	   } 
	 & & \\ 


    \theutterance \stepcounter{utterance}

    & & & \multicolumn{2}{p{0.3\linewidth}}{\cellcolor[rgb]{0.95,0.95,0.95}{%
	\makecell[{{p{\linewidth}}}]{% 
	  \tt {\tiny [GM$|$GM]}  
	 Aňlatma:\\ \tt \\ \tt Maksat gözenegi, beýleki iki daşlaşdyryjy gözenekden tapawutlanýar. Ol, maksatly gözenekdi we olaryň ortasynda doly öýjükler ("X") bar. Bu gözenekde, maksatly öýjükler ("X") gözeneklerin ortasynda ýerleşip, olaryň ululygy 
	  } 
	   } 
	   } 
	 & & \\ 


    \theutterance \stepcounter{utterance}

    & & \multicolumn{4}{p{0.6\linewidth}}{\cellcolor[rgb]{0.9,0.9,0.9}{%
	\makecell[{{p{\linewidth}}}]{% 
	  \tt {\tiny [GM$\rangle$B]}  
	 Size üç sany gözenek berilýär, olaryň hersiniň ululygy 5-den 5-e çenli.\\ \tt Torlarda "▢" bilen bellenen boş öýjükler we "X" bilen bellenen doly öýjükler bar.\\ \tt Şeýle hem size berlen gözenekleriň birini teswirleÿän aňlatma berilýär.\\ \tt Siziň wezipäňiz, berlen aňlatma gabat gelýän gözenegi saýlamakdyr. Berlen aňlatma gabat gelýän gözenegiň sanyny birinji, ikinji ýa-da üçünji ýaly (tekstde) dörediň.\\ \tt \\ \tt Birinji:\\ \tt \\ \tt X ▢ ▢ ▢ X\\ \tt X ▢ ▢ ▢ X\\ \tt X X X X X\\ \tt X ▢ ▢ ▢ X\\ \tt X ▢ ▢ ▢ X\\ \tt \\ \tt Ikinji:\\ \tt \\ \tt X ▢ ▢ ▢ X\\ \tt X X ▢ X X\\ \tt X ▢ X ▢ X\\ \tt X ▢ ▢ ▢ X\\ \tt X ▢ ▢ ▢ X\\ \tt \\ \tt Üçünji:\\ \tt \\ \tt X ▢ ▢ ▢ X\\ \tt X X ▢ ▢ X\\ \tt X ▢ X ▢ X\\ \tt X ▢ ▢ X X\\ \tt X ▢ ▢ ▢ X\\ \tt \\ \tt Aňlatma:\\ \tt \\ \tt Maksat gözenegi, beýleki iki daşlaşdyryjy gözenekden tapawutlanýar. Ol, maksatly gözenekdi we olaryň ortasynda doly öýjükler ("X") bar. Bu gözenekde, maksatly öýjükler ("X") gözeneklerin ortasynda ýerleşip, olaryň ululygy\\ \tt Sorag: Aňlatma haýsy seti aňladýar?\\ \tt "Jogap:" belligi bilen başlap saýlanan sany ýazyň. Başga tekst döretme. 
	  } 
	   } 
	   } 
	 & \\ 


    \theutterance \stepcounter{utterance}

    & & & \multicolumn{4}{p{0.6\linewidth}}{\cellcolor[rgb]{1,0.85,0.72}{%
	\makecell[{{p{\linewidth}}}]{% 
	  \tt {\tiny [GM$\langle$B]}  
	 Üçünji. 
	  } 
	   } 
	   } 
	  \\ 


    \theutterance \stepcounter{utterance}

    & & & \multicolumn{2}{p{0.3\linewidth}}{\cellcolor[rgb]{0.95,0.95,0.95}{%
	\makecell[{{p{\linewidth}}}]{% 
	  \tt {\tiny [GM$|$GM]}  
	 Invalid generated choice 
	  } 
	   } 
	   } 
	 & & \\ 


\end{supertabular}
}

\end{document}
