
\documentclass{article}
\usepackage{colortbl}
\usepackage{makecell}
\usepackage{multirow}
\usepackage{supertabular}

\begin{document}

\newcounter{utterance}

\twocolumn

{ \footnotesize  \setcounter{utterance}{1}
\setlength{\tabcolsep}{0pt}
\begin{supertabular}{c@{$\;$}|p{.15\linewidth}@{}p{.15\linewidth}p{.15\linewidth}p{.15\linewidth}p{.15\linewidth}p{.15\linewidth}}

    \# & $\;$A & \multicolumn{4}{c}{Game Master} & $\;\:$B\\
    \hline 

    \theutterance \stepcounter{utterance}  

    & & \multicolumn{4}{p{0.6\linewidth}}{\cellcolor[rgb]{0.9,0.9,0.9}{%
	\makecell[{{p{\linewidth}}}]{% 
	  \tt {\tiny [A$\langle$GM]}  
	 5x5 పరిమాణంలో ఉన్న మూడు గ్రిడ్‌లు ఉన్నాయి. ప్రతీ గ్రిడ్‌లోనూ "▢" ఖాళీ గడిని, "X", నింపిన గడిని సూచిస్తుంది.\\ \tt మూడు గ్రిడ్‌లకూ కొన్ని పోలికలు ఉండచ్చు. అయితే, మొదటి గ్రిడ్ మాత్రమే ప్రధానం, మిగిలిన రెండు గ్రిడ్‌లు దృష్టి మరల్చడానికి వాడబడ్డాయి.\\ \tt \\ \tt మీరు చేయవలసిందేంటంటే, ప్రధానమైన గ్రిడ్‌ను విపులంగా వివరించడం.\\ \tt \\ \tt \\ \tt ప్రధానమైన గ్రిడ్:\\ \tt \\ \tt X ▢ ▢ ▢ ▢\\ \tt X X ▢ ▢ ▢\\ \tt X X X ▢ ▢\\ \tt X X X X ▢\\ \tt X X X X X\\ \tt \\ \tt రెండవ గ్రిడ్:\\ \tt \\ \tt X ▢ ▢ ▢ ▢\\ \tt ▢ X ▢ ▢ ▢\\ \tt ▢ ▢ X ▢ ▢\\ \tt ▢ ▢ ▢ X ▢\\ \tt ▢ ▢ ▢ ▢ X\\ \tt \\ \tt మూడవ గ్రిడ్:\\ \tt \\ \tt X ▢ ▢ ▢ X\\ \tt ▢ X ▢ X ▢\\ \tt ▢ ▢ X ▢ ▢\\ \tt ▢ X ▢ X ▢\\ \tt X ▢ ▢ ▢ X\\ \tt \\ \tt సూచన: ప్రధానమైన గ్రిడ్‌ను వివరించుము.\\ \tt మీ ప్రత్యుత్తరాన్ని, "వివరణ: " తో మొదలుపెట్టండి. మీ ప్రత్యుత్తరంలో కేవలం ప్రధానమైన గ్రిడ్‌కు సంబంధించిన వివరణ మాత్రమే ఉండాలి. ఇతర విషయాలు ప్రస్తావించవద్దు. 
	  } 
	   } 
	   } 
	 & \\ 
 

    \theutterance \stepcounter{utterance}  

    & \multicolumn{4}{p{0.6\linewidth}}{\cellcolor[rgb]{0.8,1,0.9}{%
	\makecell[{{p{\linewidth}}}]{% 
	  \tt {\tiny [A$\rangle$GM]}  
	 వివరణ: ప్రధానమైన గ్రిడ్ 5x5 పరిమాణంలో ఉంది. ఈ గ్రిడ్‌లో, ఎడమ న 
	  } 
	   } 
	   } 
	 & & \\ 
 

    \theutterance \stepcounter{utterance}  

    & & & \multicolumn{2}{p{0.3\linewidth}}{\cellcolor[rgb]{0.95,0.95,0.95}{%
	\makecell[{{p{\linewidth}}}]{% 
	  \tt {\tiny [GM$|$GM]}  
	 వివరణ: ప్రధానమైన గ్రిడ్ 5x5 పరిమాణంలో ఉంది. ఈ గ్రిడ్‌లో, ఎడమ న 
	  } 
	   } 
	   } 
	 & & \\ 
 

    \theutterance \stepcounter{utterance}  

    & & \multicolumn{4}{p{0.6\linewidth}}{\cellcolor[rgb]{0.9,0.9,0.9}{%
	\makecell[{{p{\linewidth}}}]{% 
	  \tt {\tiny [GM$\rangle$B]}  
	 5x5 పరిమాణంలో ఉన్న మూడు గ్రిడ్‌లు ఉన్నాయి. ప్రతీ గ్రిడ్‌లోనూ "▢" ఖాళీ గడిని, "X", నింపిన గడిని సూచిస్తుంది.\\ \tt ఈ మూడు గ్రిడ్‌లతో పాటు, వీటిల్లో ఒక గ్రిడ్ వివరణ కూడ ఉంది.\\ \tt \\ \tt మీరు చేయవలసిందేంటంటే, ఆ వివరణకు సరిపోయే గ్రిడ్‌ను కనిపెట్టడం.\\ \tt \\ \tt \\ \tt మొదటిది:\\ \tt \\ \tt X ▢ ▢ ▢ X\\ \tt ▢ X ▢ X ▢\\ \tt ▢ ▢ X ▢ ▢\\ \tt ▢ X ▢ X ▢\\ \tt X ▢ ▢ ▢ X\\ \tt \\ \tt రెండవది:\\ \tt \\ \tt X ▢ ▢ ▢ ▢\\ \tt ▢ X ▢ ▢ ▢\\ \tt ▢ ▢ X ▢ ▢\\ \tt ▢ ▢ ▢ X ▢\\ \tt ▢ ▢ ▢ ▢ X\\ \tt \\ \tt మూడవది:\\ \tt \\ \tt X ▢ ▢ ▢ ▢\\ \tt X X ▢ ▢ ▢\\ \tt X X X ▢ ▢\\ \tt X X X X ▢\\ \tt X X X X X\\ \tt \\ \tt వివరణ: ప్రధానమైన గ్రిడ్ 5x5 పరిమాణంలో ఉంది. ఈ గ్రిడ్‌లో, ఎడమ న\\ \tt ప్రశ్న: ఇచ్చిన మూడు గ్రిడ్‌లలో, ఈ వివరణ దేనికి సరిపోతుంది?\\ \tt మీ ప్రత్యుత్తరాన్ని, "సమాధానం: ", తో మొదలుపెట్టండి. మీ ప్రత్యుత్తరంలో కేవలం యే గ్రిడ్ సరిపోతుందో మాత్రమే ఉండాలి. ఇతర విషయాలు ప్రస్తావించవద్దు.\\ \tt  
	  } 
	   } 
	   } 
	 & \\ 
 

    \theutterance \stepcounter{utterance}  

    & & & \multicolumn{4}{p{0.6\linewidth}}{\cellcolor[rgb]{1,0.85,0.72}{%
	\makecell[{{p{\linewidth}}}]{% 
	  \tt {\tiny [GM$\langle$B]}  
	 సమాధానం: 3 
	  } 
	   } 
	   } 
	  \\ 
 

    \theutterance \stepcounter{utterance}  

    & & & \multicolumn{2}{p{0.3\linewidth}}{\cellcolor[rgb]{0.95,0.95,0.95}{%
	\makecell[{{p{\linewidth}}}]{% 
	  \tt {\tiny [GM$|$GM]}  
	 సమాధానం: 3 
	  } 
	   } 
	   } 
	 & & \\ 
 

\end{supertabular}
}

\end{document}
