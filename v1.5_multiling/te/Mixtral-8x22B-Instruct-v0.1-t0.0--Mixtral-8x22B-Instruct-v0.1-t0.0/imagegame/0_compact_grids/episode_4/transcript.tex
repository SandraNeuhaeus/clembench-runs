
\documentclass{article}
\usepackage{colortbl}
\usepackage{makecell}
\usepackage{multirow}
\usepackage{supertabular}

\begin{document}

\newcounter{utterance}

\twocolumn

{ \footnotesize  \setcounter{utterance}{1}
\setlength{\tabcolsep}{0pt}
\begin{supertabular}{c@{$\;$}|p{.15\linewidth}@{}p{.15\linewidth}p{.15\linewidth}p{.15\linewidth}p{.15\linewidth}p{.15\linewidth}}

    \# & $\;$A & \multicolumn{4}{c}{Game Master} & $\;\:$B\\
    \hline

    \theutterance \stepcounter{utterance}

    & & \multicolumn{4}{p{0.6\linewidth}}{\cellcolor[rgb]{0.9,0.9,0.9}{%
	\makecell[{{p{\linewidth}}}]{% 
	  \tt {\tiny [A$\langle$GM]}  
	 ఈ క్రింద ఉన్న ఖాళీ గ్రిడ్‌ని పూరించడమే మన లక్ష్యం:\\ \tt \\ \tt ▢ ▢ ▢ ▢ ▢\\ \tt ▢ ▢ ▢ ▢ ▢\\ \tt ▢ ▢ ▢ ▢ ▢\\ \tt ▢ ▢ ▢ ▢ ▢\\ \tt ▢ ▢ ▢ ▢ ▢\\ \tt \\ \tt పైన ఉన్న 5 x 5 గ్రిడ్ పూరింపబడిన తరువాత ఇలా ఉంటుంది:\\ \tt \\ \tt ▢ ▢ ▢ ▢ ▢\\ \tt ▢ ▢ E ▢ ▢\\ \tt ▢ ▢ ▢ ▢ ▢\\ \tt ▢ ▢ ▢ ▢ ▢\\ \tt X X X X X\\ \tt \\ \tt మీరు ఈ గ్రిడ్‌ని నాకు దశలవారీగా వివరించాలని నేను కోరుకుంటున్నాను. ప్రతీ గ్రిడ్‌లోనూ "▢" ఖాళీ గడిని, "X", నింపిన గడిని సూచిస్తుంది. మీరు "▢" ఖాళీ గడిని వివరించాల్సిన అవసరం లేదు. గ్రిడ్‌లోని నింపిన స్థానాన్ని మాత్రమే వివరించండి. తర్వాత "మీ ​​తదుపరి సూచన ఏమిటి?" అని నేను అడిగే వరకు మీరు వేచి ఉండి, ఆపై మీరు తదుపరి దశను కొనసాగించండి. సూచనలు ఇస్తున్నప్పుడు గ్రిడ్ పరిమాణాన్ని పరిగణనలోకి తీసుకోండి. మీ వివరణ పూర్తైన తరువాత, "పూర్తయింది" అని చెప్పండి.\\ \tt \\ \tt పైన పూరింపబడిన గ్రిడ్ వివరణ, క్రింద ఉదాహరణల ద్వారా తెలుసుకోవచ్చు.\\ \tt \\ \tt మీ తదుపరి సూచన ఏమిటి?\\ \tt సూచన: రెండు అడ్డం మూడు నిలువులో Eని ఉంచండి\\ \tt \\ \tt మీ తదుపరి సూచన ఏమిటి?\\ \tt సూచన: చివరి వరుసను Xతో పూరించండి\\ \tt \\ \tt మీ తదుపరి సూచన ఏమిటి?\\ \tt సూచన: పూర్తయింది\\ \tt \\ \tt పైన ఇచ్చిన ఉదాహరణల ఆధారంగా, ఈ క్రింది గ్రిడ్‌ను వివరించండి, ఇది 5 x 5 పరిమాణం గల గ్రిడ్.\\ \tt \\ \tt ▢ R R ▢ ▢\\ \tt ▢ R R ▢ ▢\\ \tt ▢ R R ▢ ▢\\ \tt ▢ R R ▢ ▢\\ \tt ▢ R R ▢ ▢\\ \tt \\ \tt మీ తదుపరి సూచన ఏమిటి?\\ \tt  
	  } 
	   } 
	   } 
	 & \\ 


    \theutterance \stepcounter{utterance}

    & \multicolumn{4}{p{0.6\linewidth}}{\cellcolor[rgb]{0.8,1,0.9}{%
	\makecell[{{p{\linewidth}}}]{% 
	  \tt {\tiny [A$\rangle$GM]}  
	 సూచన: ప్రథమ వరుసాన్ని Rతో పూరించండి\\ \tt \\ \tt మీ తదుపరి సూచన ఏమిట 
	  } 
	   } 
	   } 
	 & & \\ 


    \theutterance \stepcounter{utterance}

    & & & \multicolumn{2}{p{0.3\linewidth}}{\cellcolor[rgb]{0.95,0.95,0.95}{%
	\makecell[{{p{\linewidth}}}]{% 
	  \tt {\tiny [GM$|$GM]}  
	 Invalid instruction format 
	  } 
	   } 
	   } 
	 & & \\ 


\end{supertabular}
}

\end{document}
