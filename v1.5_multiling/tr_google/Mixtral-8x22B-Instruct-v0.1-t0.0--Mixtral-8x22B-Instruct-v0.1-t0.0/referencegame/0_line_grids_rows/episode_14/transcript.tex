
\documentclass{article}
\usepackage{colortbl}
\usepackage{makecell}
\usepackage{multirow}
\usepackage{supertabular}

\begin{document}

\newcounter{utterance}

\twocolumn

{ \footnotesize  \setcounter{utterance}{1}
\setlength{\tabcolsep}{0pt}
\begin{supertabular}{c@{$\;$}|p{.15\linewidth}@{}p{.15\linewidth}p{.15\linewidth}p{.15\linewidth}p{.15\linewidth}p{.15\linewidth}}

    \# & $\;$A & \multicolumn{4}{c}{Game Master} & $\;\:$B\\
    \hline

    \theutterance \stepcounter{utterance}

    & & \multicolumn{4}{p{0.6\linewidth}}{\cellcolor[rgb]{0.9,0.9,0.9}{%
	\makecell[{{p{\linewidth}}}]{% 
	  \tt {\tiny [A$\langle$GM]}  
	 Size her biri 5'e 5 boyutunda olan üç ızgara veriliyor.\\ \tt Izgaralarda "▢" ile işaretlenmiş boş hücreler ve "X" ile işaretlenmiş dolu hücreler bulunur.\\ \tt Göreviniz, hedef ızgarayı en iyi şekilde tanımlayan ve onu diğer iki çeldirici ızgaradan ayıran, atıfta bulunan bir ifade oluşturmaktır.\\ \tt İlk ızgara hedef ızgaradır ve sonraki iki ızgara çeldiricilerdir.\\ \tt \\ \tt Hedef ızgarası:\\ \tt \\ \tt ▢ ▢ ▢ ▢ ▢\\ \tt ▢ ▢ ▢ ▢ ▢\\ \tt X X X X X\\ \tt ▢ ▢ ▢ ▢ ▢\\ \tt ▢ ▢ ▢ ▢ ▢\\ \tt \\ \tt Distraktör ızgarası 1:\\ \tt \\ \tt X X X X X\\ \tt ▢ ▢ ▢ ▢ ▢\\ \tt ▢ ▢ ▢ ▢ ▢\\ \tt ▢ ▢ ▢ ▢ ▢\\ \tt ▢ ▢ ▢ ▢ ▢\\ \tt \\ \tt Distraktör ızgarası 2:\\ \tt \\ \tt ▢ ▢ ▢ ▢ ▢\\ \tt X X X X X\\ \tt ▢ ▢ ▢ ▢ ▢\\ \tt ▢ ▢ ▢ ▢ ▢\\ \tt ▢ ▢ ▢ ▢ ▢\\ \tt \\ \tt Talimat: Hedef ızgarayı tanımlayın.\\ \tt Verilen hedef ızgara için "İfade: " etiketiyle başlayan başvuran ifadeyi oluşturun. Başka herhangi bir metni atlayın. 
	  } 
	   } 
	   } 
	 & \\ 


    \theutterance \stepcounter{utterance}

    & \multicolumn{4}{p{0.6\linewidth}}{\cellcolor[rgb]{0.8,1,0.9}{%
	\makecell[{{p{\linewidth}}}]{% 
	  \tt {\tiny [A$\rangle$GM]}  
	 Ifade: The target grid is the one where the bottom row is filled with "X" marks.\\ \tt \\ \tt Translation: The target grid is the one where the bottom row is filled with "X" marks.\\ \tt \\ \tt Explanation: The target grid is the one where the bottom row is filled with "X" marks, while the distractor grids have the "X" marks in different rows. This distinguishing feature allows us to identify the target grid from the other two dist 
	  } 
	   } 
	   } 
	 & & \\ 


    \theutterance \stepcounter{utterance}

    & & & \multicolumn{2}{p{0.3\linewidth}}{\cellcolor[rgb]{0.95,0.95,0.95}{%
	\makecell[{{p{\linewidth}}}]{% 
	  \tt {\tiny [GM$|$GM]}  
	 Invalid generated expression 
	  } 
	   } 
	   } 
	 & & \\ 


\end{supertabular}
}

\end{document}
