
\documentclass{article}
\usepackage{colortbl}
\usepackage{makecell}
\usepackage{multirow}
\usepackage{supertabular}

\begin{document}

\newcounter{utterance}

\twocolumn

{ \footnotesize  \setcounter{utterance}{1}
\setlength{\tabcolsep}{0pt}
\begin{supertabular}{c@{$\;$}|p{.15\linewidth}@{}p{.15\linewidth}p{.15\linewidth}p{.15\linewidth}p{.15\linewidth}p{.15\linewidth}}

    \# & $\;$A & \multicolumn{4}{c}{Game Master} & $\;\:$B\\
    \hline

    \theutterance \stepcounter{utterance}

    & & \multicolumn{4}{p{0.6\linewidth}}{\cellcolor[rgb]{0.9,0.9,0.9}{%
	\makecell[{{p{\linewidth}}}]{% 
	  \tt {\tiny [A$\langle$GM]}  
	 Давай выполним следующую задачу. Цель состоит в том, чтобы заполнить пустую сетку, которая выглядит следующим образом:\\ \tt \\ \tt ▢ ▢ ▢ ▢ ▢\\ \tt ▢ ▢ ▢ ▢ ▢\\ \tt ▢ ▢ ▢ ▢ ▢\\ \tt ▢ ▢ ▢ ▢ ▢\\ \tt ▢ ▢ ▢ ▢ ▢\\ \tt \\ \tt Заполненная сетка ниже имеет размер 5 на 5 и может выглядеть следующим образом:\\ \tt \\ \tt ▢ ▢ ▢ ▢ ▢\\ \tt ▢ ▢ E ▢ ▢\\ \tt ▢ ▢ ▢ ▢ ▢\\ \tt ▢ ▢ ▢ ▢ ▢\\ \tt X X X X X\\ \tt \\ \tt Я хочу, чтобы ты описал мне эту сетку шаг за шагом. Тебе не нужно описывать пустые квадраты, которые обозначены знаком «▢». Опиши только расположение букв в сетке. Затем подожди, пока я скажу: «Какова твоя следующая команда?», а затем переходи к следующему шагу. При составлении команд учитывай размер сетки. Когда ты все описал, ты говоришь «ГОТОВО».\\ \tt \\ \tt Вот примеры шагов для заполненной сетки выше.\\ \tt \\ \tt Какова твоя следующая команда?\\ \tt Команда: Поставь букву E во второй строке в третьем столбце.\\ \tt \\ \tt Какова твоя следующая команда?\\ \tt Команда: Заполни последнюю строку знаком X.\\ \tt \\ \tt Какова твоя следующая команда?\\ \tt Команда: ГОТОВО\\ \tt \\ \tt Хорошо. Сделай это для следующего примера, который представляет собой сетку 5 на 5.\\ \tt \\ \tt V V V V V\\ \tt ▢ ▢ ▢ V ▢\\ \tt ▢ ▢ V ▢ ▢\\ \tt ▢ V ▢ ▢ ▢\\ \tt V V V V V\\ \tt \\ \tt Какова твоя следующая команда?\\ \tt  
	  } 
	   } 
	   } 
	 & \\ 


    \theutterance \stepcounter{utterance}

    & \multicolumn{4}{p{0.6\linewidth}}{\cellcolor[rgb]{0.8,1,0.9}{%
	\makecell[{{p{\linewidth}}}]{% 
	  \tt {\tiny [A$\rangle$GM]}  
	 Команда: Поставь букву V в первом столбце во второй строке. 
	  } 
	   } 
	   } 
	 & & \\ 


    \theutterance \stepcounter{utterance}

    & & & \multicolumn{2}{p{0.3\linewidth}}{\cellcolor[rgb]{0.95,0.95,0.95}{%
	\makecell[{{p{\linewidth}}}]{% 
	  \tt {\tiny [GM$|$GM]}  
	 Команда: Поставь букву V в первом столбце во второй строке. 
	  } 
	   } 
	   } 
	 & & \\ 


    \theutterance \stepcounter{utterance}

    & & \multicolumn{4}{p{0.6\linewidth}}{\cellcolor[rgb]{0.9,0.9,0.9}{%
	\makecell[{{p{\linewidth}}}]{% 
	  \tt {\tiny [GM$\rangle$B]}  
	 Давай вместе что-нибудь нарисуем. Есть пустая сетка размером 5 на 5:\\ \tt \\ \tt ▢ ▢ ▢ ▢ ▢\\ \tt ▢ ▢ ▢ ▢ ▢\\ \tt ▢ ▢ ▢ ▢ ▢\\ \tt ▢ ▢ ▢ ▢ ▢\\ \tt ▢ ▢ ▢ ▢ ▢\\ \tt \\ \tt Я дам тебе команду типа «поставь X слева сверху», а ты вернешь сетку, следуя данной команде, например:\\ \tt \\ \tt Команда: поставь X слева сверху.\\ \tt \\ \tt X ▢ ▢ ▢ ▢\\ \tt ▢ ▢ ▢ ▢ ▢\\ \tt ▢ ▢ ▢ ▢ ▢\\ \tt ▢ ▢ ▢ ▢ ▢\\ \tt ▢ ▢ ▢ ▢ ▢\\ \tt \\ \tt Теперь создай пустую сетку размером 5 на 5 и на каждом шаге выполняй последующие команды. После выполнения команды верни только сетку и исключи весь остальной текст из вывода.\\ \tt Команда: Поставь букву V в первом столбце во второй строке. 
	  } 
	   } 
	   } 
	 & \\ 


    \theutterance \stepcounter{utterance}

    & & & \multicolumn{4}{p{0.6\linewidth}}{\cellcolor[rgb]{1,0.85,0.72}{%
	\makecell[{{p{\linewidth}}}]{% 
	  \tt {\tiny [GM$\langle$B]}  
	 В ▢ ▢ ▢ ▢\\ \tt ▢ ▢ ▢ ▢ ▢\\ \tt ▢ ▢ ▢ ▢ ▢\\ \tt ▢ ▢ ▢ ▢ ▢\\ \tt ▢ ▢ ▢ ▢ ▢ 
	  } 
	   } 
	   } 
	  \\ 


    \theutterance \stepcounter{utterance}

    & & & \multicolumn{2}{p{0.3\linewidth}}{\cellcolor[rgb]{0.95,0.95,0.95}{%
	\makecell[{{p{\linewidth}}}]{% 
	  \tt {\tiny [GM$|$GM]}  
	 Invalid grid format 
	  } 
	   } 
	   } 
	 & & \\ 


\end{supertabular}
}

\end{document}
